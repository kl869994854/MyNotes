% Options for packages loaded elsewhere
\PassOptionsToPackage{unicode}{hyperref}
\PassOptionsToPackage{hyphens}{url}
%
\documentclass[
]{article}
\usepackage{amsmath,amssymb}
\usepackage{lmodern}
\usepackage{iftex}
\ifPDFTeX
  \usepackage[T1]{fontenc}
  \usepackage[utf8]{inputenc}
  \usepackage{textcomp} % provide euro and other symbols
\else % if luatex or xetex
  \usepackage{unicode-math}
  \defaultfontfeatures{Scale=MatchLowercase}
  \defaultfontfeatures[\rmfamily]{Ligatures=TeX,Scale=1}
\fi
% Use upquote if available, for straight quotes in verbatim environments
\IfFileExists{upquote.sty}{\usepackage{upquote}}{}
\IfFileExists{microtype.sty}{% use microtype if available
  \usepackage[]{microtype}
  \UseMicrotypeSet[protrusion]{basicmath} % disable protrusion for tt fonts
}{}
\makeatletter
\@ifundefined{KOMAClassName}{% if non-KOMA class
  \IfFileExists{parskip.sty}{%
    \usepackage{parskip}
  }{% else
    \setlength{\parindent}{0pt}
    \setlength{\parskip}{6pt plus 2pt minus 1pt}}
}{% if KOMA class
  \KOMAoptions{parskip=half}}
\makeatother
\usepackage{xcolor}
\IfFileExists{xurl.sty}{\usepackage{xurl}}{} % add URL line breaks if available
\IfFileExists{bookmark.sty}{\usepackage{bookmark}}{\usepackage{hyperref}}
\hypersetup{
  hidelinks,
  pdfcreator={LaTeX via pandoc}}
\urlstyle{same} % disable monospaced font for URLs
\setlength{\emergencystretch}{3em} % prevent overfull lines
\providecommand{\tightlist}{%
  \setlength{\itemsep}{0pt}\setlength{\parskip}{0pt}}
\setcounter{secnumdepth}{-\maxdimen} % remove section numbering
\ifLuaTeX
  \usepackage{selnolig}  % disable illegal ligatures
\fi

\author{}
\date{}
\usepackage{xeCJK}
\begin{document}

\hypertarget{ux6cdbux51fd-ascoli-theorem-and-stone-weierstrass-theorem}{%
\section{泛函-Ascoli Theorem and Stone-Weierstrass
Theorem}\label{ux6cdbux51fd-ascoli-theorem-and-stone-weierstrass-theorem}}

P89

K:紧的\texttt{T\_\{2\}}空间。E:度量空间。\\
Ascoli定理考察C(K,E)中集合的紧性,Stone-Weierstrass定理考察C(K,E)上的逼近。

\hypertarget{ascoliux5b9aux7406}{%
\subsection{\texorpdfstring{\textbf{Ascoli定理:}}{Ascoli定理:}}\label{ascoliux5b9aux7406}}

\(\mathcal H\subset C(K,E)\)为相对紧的\(\Leftrightarrow\)

\begin{enumerate}
\def\labelenumi{\arabic{enumi}.}
\item
  \(\mathcal H\)等度连续。
\item
  轨道\(\mathcal H(x)=\{f(x)\,|\,\forall \, f\in \mathcal H \}\)在E中是相对紧的。
\end{enumerate}

\hypertarget{stone-weierstrassux5b9aux7406}{%
\subsection{\texorpdfstring{\textbf{Stone-Weierstrass定理:}}{Stone-Weierstrass定理:}}\label{stone-weierstrassux5b9aux7406}}

\hypertarget{ux6cdbux51fd-baire-theorem-and-its-three-applications}{%
\section{泛函-Baire Theorem and its three
applications}\label{ux6cdbux51fd-baire-theorem-and-its-three-applications}}

\hypertarget{baire-theorem}{%
\subsection{Baire Theorem}\label{baire-theorem}}

P109\\
设(E,d)是完备度量空间,\((O_{n})_{n\geq 1}\)是一列在E中稠密的开子集,则\(O=\cap_{n\geq 1}O_{n}\)在E中稠密。

\hypertarget{ux8bc1ux660e-1}{%
\subsubsection{\texorpdfstring{\textbf{证明:}}{证明:}}\label{ux8bc1ux660e-1}}

\hypertarget{ux5b9aux7406613}{%
\subsubsection{\texorpdfstring{\textbf{定理6.1.3}
}{定理6.1.3 }}\label{ux5b9aux7406613}}

局部紧的\texttt{T\_\{2\}}空间是Baire空间。\\
P110\\
\textbf{证明:}

\textbf{Notation:}

\begin{enumerate}
\def\labelenumi{\arabic{enumi}.}
\item
  Baire性质是一个拓扑性质。
\item
  Baire性质可以用闭集来等价刻画:可数个无内点的闭集的并无内点。\\
  因为,无内点闭集的补集是稠密开集。
\end{enumerate}

E是Baire空间,E中可数多个贫集的并是贫集,可数多个剩余集的交是剩余集。

\hypertarget{banach-steinhaus-theoremprinciple-of-concentration-of-singularity}{%
\subsection{Banach-Steinhaus Theorem(principle of concentration of
singularity)}\label{banach-steinhaus-theoremprinciple-of-concentration-of-singularity}}

E是Banach空间,F是赋范空间,\((u_{i})_{i \in E} \subset B(E,F)\),若满足

\[sup_{i \in I}||u_{i}(x)||<\infty \quad (\forall \, x \in E)\]

则有

\[sup_{i \in I}||u_{i}|| < \infty\]

更进一步,有

\hypertarget{ux5b9aux7406624}{%
\subsection{\texorpdfstring{\textbf{定理6.2.4}}{定理6.2.4}}\label{ux5b9aux7406624}}

设E是Banach空间,F是赋范空间,\(\{u_{n}\}_{n\geq 1}\subset B(E,F)\)。若\(\sup_{n}||u_{n}|| = \infty\),则\(\{x\in E\, |\, \sup_{n}||u_{n}(x)||=\infty\}\)为稠密的\(G_{\delta}\)集。

逆否命题:若\(\{x \in E| \sup_{n} ||u_{n}(x)||=\infty \}\)不是稠密的\(G_{\delta}\)集,则\(\sup_{n}||u_{n}||<\infty\)。\\
\(\{x \in E| \sup_{n} ||u_{n}(x)||=\infty \}\)不是稠密的\(G_{\delta}\)集\(\Leftrightarrow\)\(\{x \in E| \sup_{n} ||u_{n}(x)||<\infty \}\)不是无内点的\(F_{\sigma}\)集。\\
也就是说只要验证某点及其一个邻域内满足\(\sup_{n}||u_{n}(x)||\leq\infty\),即可得到\(\sup_{n}||u_{n}||<\infty\)。

\hypertarget{ux63a8ux8bba625}{%
\subsection{\texorpdfstring{\textbf{推论6.2.5:}}{推论6.2.5:}}\label{ux63a8ux8bba625}}

P115 \\
E是Banach空间,F是赋范空间,\((u_{i})_{i \in E} \subset B(E,F)\),且\(\forall \, x \in \, E \, , \, \lim_{n \rightarrow \infty}u_{n}(x) =u(x)\)。则有

\[u(x)\in B(E,F),||u|| \, \leq \, \underline\lim_{n\rightarrow\infty}||u_{n}||\]

\hypertarget{ux8bc1ux660e-2}{%
\subsection{\texorpdfstring{\textbf{证明:}
}{证明: }}\label{ux8bc1ux660e-2}}

根据极限的线性性,u的线性性显然。\\
\(\forall \, x\in E\),由\(u_{n}(x)\rightarrow u(x)\)得\(||u_{n}(x)||\leq\infty\),再根据B-S,得\(||u||\leq \infty\),即\(u\in B(E,F)\)。\\
进一步有:
\[
\begin{aligned}
||u(x)||&=||\lim_{n\rightarrow\infty}u_{n}(x)||\\
&=\lim_{n\rightarrow\infty}||u_{n}(x)||\\
&\leq \underline\lim_{n\rightarrow \infty}||u_{n}||\, ||x||
\end{aligned}
\]
从而结论成立。

\hypertarget{ux5f00ux6620ux5c04ux5b9aux7406}{%
\subsection{开映射定理}\label{ux5f00ux6620ux5c04ux5b9aux7406}}

\hypertarget{ux95edux56feux50cfux5b9aux7406-ux6cdbux51fd-cauch-schwartz-inequality}{%
\subsection{闭图像定理\# 泛函-Cauch-Schwartz
Inequality}\label{ux95edux56feux50cfux5b9aux7406-ux6cdbux51fd-cauch-schwartz-inequality}}

P66

\hypertarget{ux5b9aux7406-414-cauchy-schwarzux4e0dux7b49ux5f0f}{%
\subsection{\texorpdfstring{\textbf{定理 4.1.4
Cauchy-Schwarz不等式}}{定理 4.1.4 Cauchy-Schwarz不等式}}\label{ux5b9aux7406-414-cauchy-schwarzux4e0dux7b49ux5f0f}}

H是内积空间,

\[|<x,y>|^{2} \, \leq \, \,<x,x>\cdot<y,y>\quad (\forall x,y\in H)\]

等号成立当且仅当x与y成比例。

\hypertarget{ux8bc1ux660e-3}{%
\subsubsection{\texorpdfstring{\textbf{证明:}}{证明:}}\label{ux8bc1ux660e-3}}

首先考虑\(\mathbb K=\mathbb R\)的情形。\\
\(\forall \, x,y \, \in \, H\),任取\(t\, \in \, \mathbb R\),考虑\(<x+ty,x+ty>\)。\\
展开得

\[<x+ty,x+ty>=<x,x>+2t<x,y>+t^{2}<y,y> \, \geq \, 0\quad (\forall t\in\mathbb R)\]

于是得

\[|2<x,y>|^{2}-4<x,x>\cdot<y,y>\, \leq \, 0\]

即得

\[|<x,y>|^{2} \, \leq \, <x,x>\cdot<y,y>\]

再证\(\mathbb K=\mathbb C\)的情形。\\
\(\forall \, t \, \in \mathbb R\),考虑\(<x+<x,y>ty,x+<x,y>ty>\)。\\
展开得
\[
\begin{aligned}
&<x+<x,y>ty,x+<x,y>ty>\\
&=<x,x>+\overline{<x,y>t}<x,y>+<x,y>t<y,x>\\
&{\quad}+<x,y>t\cdot\overline{<x,y>t}<y,y>\\
&=<x,x>+2|<x,y>|^{2}\cdot t+|<x,y>|^{2}<y,y>\,\,\geq\,\,0
\end{aligned}
\]
类似实数情形可得到结论。

再证明等号成立的充要条件。\\
充分性显然。\\
必要性由上述证明过程可知也显然。\\
必要性:若等式成立,则由上述证明过程可知,\(\exists \, t_{0},t_{1}\in \mathbb R\)使得所考虑的\(<x+t_{0}y,x+t_{0}y=0>\)和\(<x+<x,y>t_{1}y,x+<x,y>t_{1}y>=0\)成立,从而x和y成线性。

\hypertarget{ux5e94ux7528}{%
\subsection{应用}\label{ux5e94ux7528}}

\begin{itemize}
\item
  验证了内积空间可以引出赋范空间。
\item
  离散形式通常可以验证连乘形式的范数的三角不等式。
\end{itemize}

\hypertarget{ux6cdbux51fd-ux6295ux5f71ux7684ux56dbux4e2aux57faux672cux6027ux8d28}{%
\section{泛函-投影的四个基本性质}\label{ux6cdbux51fd-ux6295ux5f71ux7684ux56dbux4e2aux57faux672cux6027ux8d28}}

P69

\hypertarget{ux6295ux5f71ux7684ux56dbux4e2aux57faux672cux6027ux8d28}{%
\subsection{\texorpdfstring{\textbf{投影的四个基本性质}}{投影的四个基本性质}}\label{ux6295ux5f71ux7684ux56dbux4e2aux57faux672cux6027ux8d28}}

H是一个Hilbert空间,C是H的一个非空闭凸集,则有:

\begin{enumerate}
\def\labelenumi{\arabic{enumi}.}
\item
  投影\(P_{C}(x)\)存在且唯一。
\item
  \(y\, \in \, C\),有\(y=P_{C}(x) \Leftrightarrow Re<x-y,z-y>\leq0 ,\, \forall z\in C\)。
\item
  \(P_{C}(x)\)为一个系数为一的Lipschitz映射,即满足\(||P_{C}(x)-P_{C}(y)||\leq ||x-y||\)。
\item
  \(P_{C}(x)\)是一个幂等映射。
\end{enumerate}

\hypertarget{ux8bc1ux660e-4}{%
\subsubsection{\texorpdfstring{\textbf{证明:}}{证明:}}\label{ux8bc1ux660e-4}}

1.\\
\(\forall \, x\in H\),记\(d=inf_{y \in C}d(x,y)\),由下确界的定义,可得一列点列\(\{y_{n}\}\subset C\),满足\(||x-y_{n}||\leq d+\frac{1}{n}\quad(\forall n\in N^{\star})\),接下来考虑放缩:
\[
\begin{aligned}
(d+\frac{1}{n})^{2}+(d+\frac{1}{m})^{2} \, & \geq \, ||x-y_{n}||^{2}+||x-y_{m}||^{2} \\
&=\frac{1}{2}(||2x-y_{n}-y_{m}||^{2}+||y_{m}-y_{n}||^{2})\\
&=2||x-\frac{y_{n}+y_{m}}{2}||^{2}+\frac{1}{2}||y_{m}-y_{n}||^{2}\\
&\geq 2d^{2}+\frac{1}{2}||y_{m}-y_{n}||^{2}
\end{aligned}
\]
则验证了\(\{y_{n}\}\)为Cauchy列,由Hilbert空间中闭集的性质,可知其收敛,记之为\(y=\lim_{n\rightarrow\infty}y_{n}\,\in\,C\),y满足\(d(x,y)=\lim_{n\rightarrow\infty}d(x,y_{n})=d\)。\\
再证唯一性。\\
如果有\(y_{1},y_{2}\, \in \, C\)满足\(d=d(x,y_{1})=d(x,y_{2})\),有类似放缩:
\[
\begin{aligned}
d^{2}+d^{2}&=||x-y_{1}||^{2}+||x-y_{2}||^{2}\\
&=\frac{1}{2}(||2x-y_{1}-y_{2}||^{2}+||y_{2}-y_{1}||^{2})\\
&=2d^{2}+\frac{1}{2}||y_{2}-y_{1}||^{2}
\end{aligned}
\]
因此\(y_{1}=y_{2}\)。

2.\\
必要性\\
Keypoint:\(\forall z\in C\),考虑\(<x-(\lambda y+(1-\lambda)z),·>\)。\\
充分性\\
显然\\
具体细节课本。

3.
\[
\begin{aligned}
||P_{C}(x)-P_{C}(y)||^{2}&=<P_{C}(x)-x+x-y+y-P_{C}(y),P_{C}(x)-P_{C}(y)>\\
&=<P_{C}(x)-x,P_{C}(x)-P_{C}(y)>+<x-y,P_{C}(x)-P_{C}(y)>\\
&{\quad}+<y-P_{C}(y),P_{C}(x)-P_{C}(y)>\\
&\leq Re<x-y,P_{C}(x)-P_{C}(y)>\\
&\leq ||x-y||\cdot||P_{C}(x)-P_{C}(y)||
\end{aligned}
\]
结论得证。

4.\\
显然。

\hypertarget{ux6cdbux51fd-ux62d3ux6251ux5411ux91cfux7a7aux95f4ux7684ux57faux672cux6027ux8d28}{%
\section{泛函-拓扑向量空间的基本性质}\label{ux6cdbux51fd-ux62d3ux6251ux5411ux91cfux7a7aux95f4ux7684ux57faux672cux6027ux8d28}}

P129

\hypertarget{ux4e00ux4e2aux5c0fux7ed3ux8bba}{%
\subsubsection{\texorpdfstring{\textbf{一个小结论}}{一个小结论}}\label{ux4e00ux4e2aux5c0fux7ed3ux8bba}}

给定拓扑空间E,及其上的连续映射\(\phi\)。\(\forall \, A\, \subset \, E\),有

\[\phi(\bar{A})\subset\overline{\phi(A)}\]

\textbf{证明}:\\
对\(\forall \phi(x)\in \phi(\bar{A})\)以及\(\forall \, V_{\phi(x)}\in \mathcal N(\phi(x))\)。\\
有\(\phi^{-1}(V_{\phi(x)}) \cap A\neq \emptyset\),即\(V_{\phi(x)} \cap \phi (A) \neq \emptyset\),即\(\phi(x)\in \overline{\phi(A)}\)。

\end{document}
