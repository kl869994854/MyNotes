\section{Hahn-Banach定理,弱拓扑和弱*拓扑}
\subsection{Hahn-Banach定理:分析形式}

\begin{definition}[次线性泛函]
	设$E$是数域$\mathbb{R}$上的向量空间,若泛函$p:\, E\rightarrow \, \mathbb{R}$满足:
	\begin{enumerate}
		\item 正齐性:
		\item 次可加性:
	\end{enumerate}
	则称该泛函$p$为$E$上的次线性泛函。
\end{definition}

\begin{lemma}
	设$E$是实向量空间,$F$是$E$的余维为1的向量子空间,并设$p:E\, \rightarrow \, \mathbf{R}$是次线性泛函,$f:F\, \rightarrow \, \mathbb{R} $是线性泛函。若在$F$上有$f \leq p $,则存在线性泛函$\tilde{f}:E\rightarrow \mathbb{R} $,使得
	\begin{equation*}
		\tilde{f}\big|_{F}=f\quad \text{且} \quad \tilde{f}(x)\leq p(x),\forall \, x\in E.
	\end{equation*}
\end{lemma}

\begin{theorem}[Hahn-Banach延拓定理:实情形]
	设$E$是实向量空间,$F\subset E $是向量子空间,并设$p\, \rightarrow \, \mathbb{R} $是次线性泛函,$f:\, F\, \rightarrow \, \mathbb{R} $是线性泛函且在$F$上满足$f\leq p$。那么总存在$f$的线性延拓$\tilde{f}: E\, \rightarrow \, \mathbb{R} $,使得
	\begin{equation*}
		\tilde{f}\big|_{F}=f\quad \text{且} \quad \tilde{f}(x)\leq p(x),\forall \, x\in E.
	\end{equation*}
\end{theorem}

\begin{theorem}[Hahn-Banach延拓定理]
	设$E$是数域$\mathbb{R}$($\mathbb{K}$为$\mathbb{R}$或$\mathbb{C}$)上的向量空间,$F\subset E $是向量子空间,并设$p$是$E$上的半范数,$f:\, F\, \rightarrow \, \mathbb{K} $是线性映射且在$F$上满足$|f|\leq p$。那么总存在线性泛函$\tilde{f}:E\,\rightarrow \, \mathbb{K} $,使得
	\begin{equation*}
		\tilde{f}\big|_{F}=f\quad \text{且} \quad \tilde{f}(x)\leq p(x),\forall \, x\in E.
	\end{equation*}
\end{theorem}

\original
{
	设$E$是复向量空间,对任意复线性泛函$f:\, E\,\rightarrow\, \mathbb{C} $,定义$\phi(f)=Re\, f $。\underline{由于线性泛函$f$的虚部可由其实部确定},则映射$\phi$建立了$E$上所有复线性泛函构成的集合到$E$上所有实线性泛函构成的集合上的双射,并且
	\begin{equation*}\tag{+}
		f(x)=\phi (f)(x)-\mathrm{i} \, \phi (f)(\mathrm{i} x),\quad \forall x\in E.
	\end{equation*}
}{P153}

\begin{proposition}
	线性泛函$f$的虚部如何由其实部确定?
\end{proposition}
\begin{proof}
	参考式子(+)。记$f(x)=\Phi(f)\,(x)+\mathrm{i}\Psi(f)\,(x) $,这里$\Phi(f),\Psi(f)$均为实函数,由线性性$f(\mathrm{i}x)=if(x)=\mathrm{i}\Phi(f)\, (x) -\Psi(f) \, (x)= \Phi(f)\, (\mathrm{i}x) +\mathrm{i}\Psi(f)\, (\mathrm{i}x) $,对照得$\Psi(f)\,(\mathrm{i}x)=\Phi(f)\,(x) $,即得$\Psi(f)\, (x)=-i\Phi(f)\, (\mathrm{i}x) $。
\end{proof}

\begin{corollary}
	设$E$是拓扑向量空间,$F$是$E$的向量子空间。并设$p:\,E \, \rightarrow \, \mathbb{K} $是连续的半范数,$f:F\, \rightarrow \, \mathbb{K} $是线性泛函且在$F$上满足$|f|\leq p $。那么存在$f$的连续线性延拓$\tilde{f}: E\, \rightarrow \, \mathbb{K} $,使得$\tilde{f\big| }_{F} =f $且在$E$上$|\tilde{f}|\leq p $。
\end{corollary}

\original
{
	在上一结论的证明中,我们注意到如下事实:如果$p,q$都是拓扑向量空间$E$上的半范数且满足$q\leq p$,那么当$p$连续时,$q$也是连续的。特别地,若$p$是$E$上的连续半范数,$f$是$E$上的线性泛函且满足$|f|\leq p$,则$f$连续。
}{P153}

\begin{proposition}
	\begin{enumerate}
		\item 验证:如果$p,q$都是拓扑向量空间$E$上的半范数且满足$q\leq p$,那么当$p$连续时,$q$也是连续的。
		\item 验证:若$p$是$E$上的连续半范数,$f$是$E$上的线性泛函且满足$|f|\leq p$,则$f$连续。
	\end{enumerate}
\end{proposition}
\begin{proof}
	\begin{enumerate}
		\item 
		\item 由线性性,$f$连续等价于$f$在$0$点连续,从而结论成立。
	\end{enumerate}
\end{proof}

\begin{corollary}
	设$E$是局部凸空间,$F\,\subset \, E $是向量子空间而$f\,: \,F\, \rightarrow \, \mathbb{K} $是连续线性泛函。那么存在$f$的连续线性延拓$\tilde{f}:\,E\,\rightarrow \, \mathbb{K} $。
\end{corollary}

\begin{corollary}
	设$E$是拓扑向量空间,$p$是$E$上的连续半范数,并设$x_{0}\in E$。那么存在$f\in E^{*} $,使得$f(x_{0})=p(x_{0}) $且在$E$上有$|f|\leq p $。
\end{corollary}

\begin{corollary}\label{XQHcor080104}
	设$E$是Hausdorff局部凸空间,则$E^{*}$是可分点的,即对于任一非零向量$x_{0} \in E$,存在$f\in E^{*} $,使得$f(x_{0})\neq 0 $。
\end{corollary}

\original
{
	推论\ref{XQHcor080104}告诉我们,Hausdorff局部凸空间上的非平凡线性泛函不仅存在,而且足够多,甚至是可分点的。
}
{P154}

\begin{proposition}
	关于可分点的定义:能把$x\neq 0$与$0$,是否等价于,能把$x_{1}\neq x_{2} $两点分开。
\end{proposition}

\begin{corollary}{XQHcor080105}
	设$E$是数域$\mathbb{K}$上的赋范空间,$F\subset E $是向量子空间且$f:F\, \rightarrow \, \mathbb{K} $是连续线性泛函。那么存在连续的线性延拓$\tilde{f} : E\, \rightarrow \, \mathbb{K} $,使得$\tilde{f}\big|_{F}=f  $且$||\tilde{f}||=||f|| $。这样的$\tilde{f} $称为$f $的保范延拓。
\end{corollary}

\begin{corollary}\label
	设$E$是赋范空间,$x_{0}\in E $,$x_{0}\neq 0 $,则存在$f\in E^{*} $,使得$f(x_{0}) =||x_{0}|| $且$||f||\leq 1 $。
\end{corollary}

\begin{proof}
	考虑$E$的线性子空间$F=\{k\, x_{0}\big| k\in \mathbb{K} \} $,与连续线性泛函
	\begin{equation*}
		{\large\text{$f:$}}\,\,\,\,
		\begin{aligned}
			F& \rightarrow \, \mathbb{K}\\
			kx_{0}& \rightarrow \, k\, ||x_{0}||
		\end{aligned}
	\end{equation*}
	这里$f(x_{0})=||x_{0}||$且$||f||\leq 1$。由推论\ref{XQHcor080105},则$f$可以延拓到$E$上的连续线性泛函$\tilde{f} $且满足$||f||=||\tilde{f}|| $,

\end{proof}

\question{为什么不直接写$||f||=1$}

\begin{corollary}
	设$E$是赋范空间,则任取$x\in E $,有
	\begin{equation*}
		||x||=\sup \{|f(x)|:\, f\in E^{*},||f||\leq 1 \}.
	\end{equation*}
	并且上确界是可以达到的。
\end{corollary}















