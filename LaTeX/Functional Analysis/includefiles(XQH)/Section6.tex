\section{Baire定理及其应用}
\subsection{Baire空间}

\begin{theorem}[Baire]
    设$(E,d)$是完备度量空间,$(O_n)_{n\geq1}$是一列在$E$中稠密的开子集,则$O=\bigcap_{n\geq1}O_n$在$E$中稠密。
\end{theorem}

\begin{definition}
    称拓扑空间$E$是一个Baire空间,若$E$中任意可数多个稠密开集的交仍然在$E$中稠密。
\end{definition}

\begin{theorem}
    局部紧的Hausdorff空间是Baire空间。
\end{theorem}

\begin{theorem}
    设$E$是$Baire$空间,则
    \begin{enumerate}
        \item $E$的任意开子集也是一个Baire空间。
        \item 设$(F_n)_{n\geq1}$是$E$的一列闭子集,并且$E=\bigcup_{n\geq1}F_n $,那么$\bigcup_{n\geq1}\mathring{F}_n$在$E$中稠密。
    \end{enumerate}
\end{theorem}

\begin{definition}
    设$E$是拓扑空间。
    \begin{enumerate}
        \item 称$E$中可数多个开子集的交集为$G_{\delta}$集,称$E$中可数多个闭子集的并集为$F_{\sigma}$集。
        \item 称$A\subset E$为贫集(或第一纲集),若$A$为某个无内点的$F_{\sigma }$集的子集;称$A\subset E$为剩余集,若$A$包含一个稠密的$G_{\delta }$集。
    \end{enumerate}
\end{definition}

\begin{theorem}
    设$E$是Baire空间,$(F,\delta )$是度量空间。并设映射序列$(f_n)_n\subset C(E,F)$逐点收敛到函数$f$。那么$f$的所有连续点构成的集合$Cont(f)$是$E$中稠密的$G_{\delta }$集。
\end{theorem}

\begin{lemma}
    
\end{lemma}

\subsection{Banach-Steinhaus定理}

\begin{theorem}[Banach-Steinhaus]
    设$E$是Banach空间,$F$是赋范空间,$(u_i)_{i\in I}$是一族从$E$到$F$的有界线性算子,即$(u_i)_{i\in I}\subset B(E,F)$。若对每一点$x\in E$,有$\sup _{i\in I}||u_i(x)||<\infty $,则
    \begin{equation*}
        \sup_{i\in I}||u_i||<\infty , 
    \end{equation*}
    即算子族$(u_i)_{i\in I}$在$B(E,F)$中有界。       
\end{theorem}

\subsection{开映射和闭图像定理}

\begin{theorem}[(开映射定理)]
    设$E$和$F$都是Banach空间,$u\in B(E,F)$。若像集$u(E)$不是$F$中的贫集,则
    \begin{enumerate}
        \item 存在$r>0$,使得$rB_F\subset u(B_E)$,这里$B_E$和$B_F$分别是$E$和$F$中的开单位球。并由此可得$u$是满射,即$u(E)=F$。
        \item $u$是开映射。
    \end{enumerate}
\end{theorem}

\begin{corollary}[(开映射定理)]
    设$E$和$F$都是Banach空间,$u\in B(E,F)$是满射,则$u$是开映射并且存在$r>0$,使得$rB_F\subset u(B_E)$。
\end{corollary}

\begin{theorem}[(闭图像定理)]
    设$E$和$F$都是Banach空间,$u:E\rightarrow F$是线性映射,则$u$连续当且仅当图像$G(u)$是闭的。
\end{theorem}