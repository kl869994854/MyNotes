\section{Banach空间的对偶理论}
\subsection{共轭算子}

\begin{theorem}
    设$E$和$F$是赋范空间,$u\in B(E,F)$。则存在唯一的$u^{*}\in B(E^{*},F^{*})$,使得对任意$x\in E$和$f^{*}\in F^{*}$,有$<u^{*}(f^{*}),x>=<f^{*},u(x)>$。并且$||u^{*}||=||u||$。我们称$u^{*}$是$u$的共轭算子。
\end{theorem}

\original
{
    \begin{enumerate}
        \item 上面定理意味着映射$u\mapsto u^{*}$是$B(E,F)$到$B(F^{*},E^{*})$的线性等距映射,但可能不是满射。例如假设$F$不完备,这种情形下$u\mapsto u^{*}$不是满射。
        \item 设$E,F,G$都是赋范空间且$u\in B(E,F)$,$v\in B(F,G)$,则$(vu)^{*}=u^{*}v^{*}$。
        \item 若$E,F$都是数域$\mathbb{K}$上的Hilbert空间,\uline{我们要注意第四章在Hilbert空间意义下定义的共轭和这里Banach空间定义的共轭存在细微的差别。}当数域$\mathbb{K}=\mathbb{R}$时,两者定义的共轭是一致的;但当$\mathbb{K}=\mathbb{C}$时,映射$u\mapsto u^{*}$在Hilbert空间下定义时共轭线性的,而在Banach空间定义下是线性的。
    \end{enumerate}
}
{P174}
\begin{proposition}
    
\end{proposition}


\begin{theorem}
    设$E$和$F$是赋范空间,$u\in B(E,F)$。
    \begin{enumerate}
        \item 当$u$时同构映射时,$u^{*}$也是同构映射,且有$(u^{*})^{-1}=(u^{-1})^{*}$。
        \item 若还假设$E$是完备的,则当$u^{*}$是同构映射时,$u$也是同构映射。
    \end{enumerate}
\end{theorem}

\subsection{子空间和商空间的对偶}

\begin{theorem}[可能的考点]
    设$E$和$F$都是赋范空间,并有$u\in \mathcal{B}(E,F)$。则
    \begin{enumerate}
        \item $\ker u^{*}=u(E)^{\perp}$.
        \item $\ker u=u^{*}(F^{*})_{\perp}$.
        \item $(\ker u^{*})_{\perp}=\overline{u(E)}^{w}=\overline{u(E)}^{||\cdot||}$.
        \item $(\ker u)^{\perp}=\overline{u^{*}(F^{*})}^{w^{*}}$.
    \end{enumerate}
\end{theorem}

\begin{proof}
    \begin{enumerate}
        \item 由赋范空间中共轭算子的定义,$\forall \, x\in E,f^{*}\in F^{*}$,均有
        \begin{equation*}
            <f^{*},u(x)>=<u^{*}(f^{*}),x>
        \end{equation*}
        则有
        \begin{equation*}
            \begin{aligned}
                f^{*}\in \ker u^{*}&\Leftrightarrow u^{*}(f^{*})(x)=0,\,\forall \,x\in E\\
                &\Leftrightarrow <u^{*}(f^{*}),x>=0\\
                &\Leftrightarrow <f^{*},u(x)>=0\\
                &\Leftrightarrow f^{*}\in u(E)^{\perp}
            \end{aligned}
        \end{equation*}
        所以$\ker u^{*}=u(E)^{\perp }$。
        \item 类似可得:
        \begin{equation*}
            \begin{aligned}
            x\in \ker u&\Leftrightarrow <f^{*},u(x)>=0,\forall f^{*}\in F^{*}\\
            &\Leftrightarrow <u^{*}(f^{*}),x>=0,\forall f^{*}\in F^{*}\\
            &\Leftrightarrow x\in u^{*}(F^{*})_{\perp}     
            \end{aligned}
        \end{equation*}
        所以$\ker u=u^{*}(F^{*})_{\perp}$。
        \item 由双极定理\ref{XQHthe080302}可得
        \begin{equation*}
            (\ker u^{*})_{\perp}=(u(E)^{\perp})_{\perp}=\overline{u(E)}^{w}=\overline{u(E)}^{||\cdot||}
        \end{equation*}
        \item 同样由双极定理\ref{XQHthe080302}得
        \begin{equation*}
            (\ker u)^{\perp}=(u^{*}(F^{*})_{\perp})^{\perp}=\overline{u^{*}(F^{*})}^{w^{*}}
        \end{equation*}
    \end{enumerate}
\end{proof}

\subsection{自反性}

\begin{example}
    一些自反空间的例子。
    \begin{enumerate}
        \item 若$dim E<+\infty$,则$E$是自反的。
        \item 所有的Hilbert空间是自反的。分别考虑实和复的情形。
    \end{enumerate}
\end{example}

\begin{proof}
    \begin{enumerate}
        \item 
        \item 先证明实情形,由Riesz表示定理,可以找到$H$和$H^{*}$之间的线性等距同构,
        \begin{equation*}
            \text{\large$\Phi:$\,\,}
            \begin{aligned}
              H&\rightarrow H^{*},\\
              y&\mapsto \Phi(y)=\phi_{y}(x)  
            \end{aligned}
        \end{equation*}
        这里$\phi_{y}(x)=<x,y>$,考虑$\Phi$的共轭算子:
        \begin{equation*}
            \text{\large$\Phi^{*}:$\,\,}
              H^{**}\rightarrow H^{*},
        \end{equation*}
        再考虑$(\Phi^{*})^{-1}\Phi$,其为$H\rightarrow H^{**}$的等距同构映射,且可验证$(\Phi^{*})^{-1}\Phi=\tau_{x}$,从而证明了$H$自反。
    \end{enumerate}
\end{proof}