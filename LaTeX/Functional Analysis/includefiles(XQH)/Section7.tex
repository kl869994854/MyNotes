
\section{拓扑向量空间}
\subsection{定义和性质}

\begin{definition}[拓扑向量空间]
	
\end{definition}

\begin{example}
	验证:$L_{p}(\Omega),\, 0<\, p\,\leq\, \infty$为拓扑向量空间。
\end{example}

\begin{proof}
	\begin{itemize}
		\item 首先验证$0\, < \, p\, < \, 1$的情形,此时$L_{p}(\Omega)$上的拓扑由距离$||f-g||^{p}=\int |f-g|^{p}\, \dd x$所诱导:
		\begin{enumerate}
			\item 先验证加法连续。$\forall \, f_{0},g_{0}\in \, L_{p}(\Omega) \quad and \forall \, V_{0}\in \mathcal{N}_{f_{0}+g_{0}}$,不妨记$V_{0}=B(f_{0}+g_{0},r)$,此时,考虑$f_{0}\in B_{1}=B(f_{0},\frac{r}{2})\in \mathcal{N}(f_{0})\, ,\, g_{0}\in B_{2}=B(g_{0},\frac{r}{2})\in\mathcal{N}(g_{0})$,则有$\forall \, f\in \, B_{1}\, ,\, g \, \in \, B_{2}\, ,\, f+g\in V_{0}$,所以加法连续。
			\item 再验证数乘成立。$\forall \, k\in\mathbb{K}\, ,f_{0}\in L_{p}(\Omega)$以及$\forall \, B=B(kf_{0},r)$,要找到开球$k\in B_{1}=B(k,r_{1})\in \mathcal{N}(k),f_{0}\in  B_{2}=B(f_{0},r_{2})\in \mathcal{N}(f_{0})$,使得$\forall \lambda\, \in \, B_{1}\, , \, g\in B_{2}$,满足:
			\begin{equation*}
				\int |\lambda g-kf_{0}|^{p}\, \dd x\, < \, r
			\end{equation*}
		利用插值技巧:
			\begin{equation*}
				\begin{aligned}
				\int |\lambda g-kf_{0}|^{p}\,\dd
				 x \, &=\int |\lambda g-\lambda f_{0} + \lambda f_{0}-kf_{0}|^{p}\,\dd x\\
				 &\leq \int |\lambda g-\lambda f_{0}|^{p}\,\dd x +\int |\lambda f_{0}-kf_{0}|^{p}\, \dd x\\
				 &\leq |\lambda|^{p}\, \int |g-f_{0}|^{p}\,\dd x \,+\, |\lambda -k|^{p} \int |f_{0}|^{p}\,\dd x \, < \,r 			 
				\end{aligned}
			\end{equation*}
		可取$r_{1}=\frac{r^{\frac{1}{p}}}{2^{\frac{1}{p}}|f_{0}|}\, ,\, r_{2}=\frac{r}{2|\lambda|^{p}_{max}}$,所以数乘也连续。
		\end{enumerate}
	\item 再验证$1\, \leq \, p \, \leq \, \infty $的情形,此时$L_{p}(\Omega)$上的拓扑是由范数$||f||=\Big(\int |f|^{p}\,\dd x\Big)^{\frac{1}{p}}$诱导。
	\begin{enumerate}
		\item 先验证加法连续,对于$\forall \, f_{0},g_{0}\in L_{p}(\Omega)$,以及$\forall \, B(f_{0}+g_{0},r)\in \, \mathcal{N}(f_{0}+g_{0})$,考虑$B(f_{0},\frac{r}{2})\, , \, B(g_{0},\frac{r}{2})$即可,所以加法连续。
		\item 再验证数乘连续,对于$\forall k\, \in \mathbb{K}\, , \, f_{0}\in \, L_{p}(\Omega)$,以及$\forall \, B=B(kf_{0},r)\, \in \, \mathcal{N}(kf_{0})$,要找$B_{1}=B(k,r_{1})\, ,\, B_{2}=B(f_{0},r_{2})$,使得$\forall \, \lambda\in B_{1}\, ,\, g\in B_{2}$,满足:
		\begin{equation*}
			(\int \, |\lambda g-kf_{0}|^{p})^{\frac{1}{p}}\, \dd x \, <\, r
		\end{equation*}
	可先考虑:
	\begin{equation*}
		\begin{aligned}
				(\int \, |\lambda g-kf_{0}|^{p})^{\frac{1}{p}}&=(\int \, |\lambda g-\lambda f_{0}+\lambda f_{0}-kf_{0}|^{p}\dd x)^{\frac{1}{p}}\\
				&\leq |\lambda|(\int |g-f_{0}|^{p}\dd x)^{\frac{1}{p}}+|\lambda-k|\cdot||f_{0}||_{p}\, <\, r
		\end{aligned}
	\end{equation*}
	由上式可取$r_{1}=\frac{r}{2||f_{0}||_{p}}\,,\, r_{2}=\frac{r}{2\lambda_{max}}$,所以数乘连续。
	\end{enumerate}
	\end{itemize}
\end{proof}

\begin{remark}
	\begin{itemize}
		\item 当$0\, <\, p\, <\, 1$时,$L_{p}(\Omega)$不是赋范空间。
		\item 当$1\, \leq p \leq \infty$,$L_{p}(\Omega)$是赋范空间。
	\end{itemize}
\end{remark}

\begin{theorem}
	设$E$是拓扑向量空间,则“伸缩平移”是$E$上的同胚。
\end{theorem}

\begin{hint}
	连续映射的复合仍是连续映射。
\end{hint}

\begin{definition}
	\begin{enumerate}
		\item 平衡的:若任取$|\lambda|\leq 1(\lambda\in \mathbb{K}) $,对$\forall x\in A$,有$\lambda x\in A $,那么称$A$是平衡的。
		\item 吸收的:对$\forall x\in E $,$\exists a>0 (constant)$,当取$|\lambda|\leq a(\lambda \in \mathbb{K})$时,有$\lambda \, x\in A$,那么称$A$是吸收的。
	\end{enumerate}
\end{definition}

\begin{remark}
	若$A$是平衡的,则$A$是对称的,即$-A=A$,且$0\in A$。
\end{remark}

\begin{theorem}
	若$E$是拓扑向量空间,$A\subset E$。则
	\begin{enumerate}
		\item 若$A$是向量子空间(凸集或平衡集),则$\bar{A}$也是向量子空间(凸集或者平衡集)。
		\item 若$A$是凸集,则$\mathring{A}$也是凸集。
		\item 若$A$是平衡集且$0\in \mathring{A} $,则$\mathring{A}$也是平衡集。
	\end{enumerate}
\end{theorem}

\begin{proof}
	\begin{enumerate}
		\item 
		\item
		\item
	\end{enumerate}
\end{proof}

接下来用$\mathcal{N}(x)$表示$x$点的领域系。
\begin{theorem}
	设$E$是拓扑向量空间。则
	\begin{enumerate}
		\item 任取$x\in E $,则$\mathcal{N}(x)=\mathcal{N}(x)+x $(即$V\in \mathcal{N}(x)\Leftrightarrow V-x\in \mathcal{N}(0) $)。
		\item 任取$V\in \mathcal{N}(0) $,存在$U\subset \mathcal{N}(0) $,使得$U+U\subset V $。
		\item 设$\lambda\in \mathbb{K} $且$\lambda\neq 0 $。则$V\in \mathcal{N}(0)\Leftrightarrow \lambda V\in \mathcal{0} $。
		\item 任意$V\in \mathcal{N}(0) $是吸收的。
		\item 原点0有平衡的开(或闭)邻域基。
	\end{enumerate}
\end{theorem}
\begin{proof}
	
\end{proof}


\begin{theorem}\label{XQHthe070104}
	设$E$和$F$是两个拓扑向量空间,$u:\, E\, \rightarrow \, F $是线性映射。则以下命题等价:
	\begin{enumerate}
		\item $u$是连续映射。
		\item $u$在原点连续。
	\end{enumerate}
	若$F$还是赋范空间,则以上命题与如下命题等价:
	\begin{enumerate}
		\item[3.] $u$在原点的某邻域内有界。 
	\end{enumerate}
\end{theorem}

\begin{proof}
	
\end{proof}

\original
{定理\ref{XQH070104}意味着连续线性映射$u:E\, \rightarrow \, F $是一致连续的,即对$F$中原点的每个邻域$V$,存在$E$中原点的邻域$W$,使得当$x-y\in W $时,$u(x)-u(y) \in V $
}
{P133}
\begin{remark}
	类比函数的一致连续理解。
\end{remark}

\begin{corollary}
	设$E$是数域$\mathcal{K}$上的拓扑向量空间,$u:E\, \rightarrow \, \mathbb{K} $是线性泛函。则定理\ref{XQH070104}的结论成立。
\end{corollary}









