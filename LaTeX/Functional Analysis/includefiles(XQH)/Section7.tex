
\section{拓扑向量空间}
\subsection{定义和基本性质}

\begin{definition}[拓扑向量空间]
	
\end{definition}

\begin{example}
	验证:$L_{p}(\Omega),\, 0<\, p\,\leq\, \infty$为拓扑向量空间。
\end{example}

\begin{proof}
	\begin{itemize}
		\item 首先验证$0\, < \, p\, < \, 1$的情形,此时$L_{p}(\Omega)$上的拓扑由距离$||f-g||^{p}=\int |f-g|^{p}\, \dd x$所诱导:
		\begin{enumerate}
			\item 先验证加法连续。$\forall \, f_{0},g_{0}\in \, L_{p}(\Omega) \quad and \forall \, V_{0}\in \mathcal{N}_{f_{0}+g_{0}}$,不妨记$V_{0}=B(f_{0}+g_{0},r)$,此时,考虑$f_{0}\in B_{1}=B(f_{0},\frac{r}{2})\in \mathcal{N}(f_{0})\, ,\, g_{0}\in B_{2}=B(g_{0},\frac{r}{2})\in\mathcal{N}(g_{0})$,则有$\forall \, f\in \, B_{1}\, ,\, g \, \in \, B_{2}\, ,\, f+g\in V_{0}$,所以加法连续。
			\item 再验证数乘成立。$\forall \, k\in\mathbb{K}\, ,f_{0}\in L_{p}(\Omega)$以及$\forall \, B=B(kf_{0},r)$,要找到开球$k\in B_{1}=B(k,r_{1})\in \mathcal{N}(k),f_{0}\in  B_{2}=B(f_{0},r_{2})\in \mathcal{N}(f_{0})$,使得$\forall \lambda\, \in \, B_{1}\, , \, g\in B_{2}$,满足:
			\begin{equation*}
				\int |\lambda g-kf_{0}|^{p}\, \dd x\, < \, r
			\end{equation*}
		利用插值技巧:
			\begin{equation*}
				\begin{aligned}
				\int |\lambda g-kf_{0}|^{p}\,\dd
				 x \, &=\int |\lambda g-\lambda f_{0} + \lambda f_{0}-kf_{0}|^{p}\,\dd x\\
				 &\leq \int |\lambda g-\lambda f_{0}|^{p}\,\dd x +\int |\lambda f_{0}-kf_{0}|^{p}\, \dd x\\
				 &\leq |\lambda|^{p}\, \int |g-f_{0}|^{p}\,\dd x \,+\, |\lambda -k|^{p} \int |f_{0}|^{p}\,\dd x \, < \,r 			 
				\end{aligned}
			\end{equation*}
		可取$r_{1}=\frac{r^{\frac{1}{p}}}{2^{\frac{1}{p}}|f_{0}|}\, ,\, r_{2}=\frac{r}{2|\lambda|^{p}_{max}}$,所以数乘也连续。
		\end{enumerate}
	\item 再验证$1\, \leq \, p \, \leq \, \infty $的情形,此时$L_{p}(\Omega)$上的拓扑是由范数$||f||=\Big(\int |f|^{p}\,\dd x\Big)^{\frac{1}{p}}$诱导。
	\begin{enumerate}
		\item 先验证加法连续,对于$\forall \, f_{0},g_{0}\in L_{p}(\Omega)$,以及$\forall \, B(f_{0}+g_{0},r)\in \, \mathcal{N}(f_{0}+g_{0})$,考虑$B(f_{0},\frac{r}{2})\, , \, B(g_{0},\frac{r}{2})$即可,所以加法连续。
		\item 再验证数乘连续,对于$\forall k\, \in \mathbb{K}\, , \, f_{0}\in \, L_{p}(\Omega)$,以及$\forall \, B=B(kf_{0},r)\, \in \, \mathcal{N}(kf_{0})$,要找$B_{1}=B(k,r_{1})\, ,\, B_{2}=B(f_{0},r_{2})$,使得$\forall \, \lambda\in B_{1}\, ,\, g\in B_{2}$,满足:
		\begin{equation*}
			(\int \, |\lambda g-kf_{0}|^{p})^{\frac{1}{p}}\, \dd x \, <\, r
		\end{equation*}
	可先考虑:
	\begin{equation*}
		\begin{aligned}
				(\int \, |\lambda g-kf_{0}|^{p})^{\frac{1}{p}}&=(\int \, |\lambda g-\lambda f_{0}+\lambda f_{0}-kf_{0}|^{p}\dd x)^{\frac{1}{p}}\\
				&\leq |\lambda|(\int |g-f_{0}|^{p}\dd x)^{\frac{1}{p}}+|\lambda-k|\cdot||f_{0}||_{p}\, <\, r
		\end{aligned}
	\end{equation*}
	由上式可取$r_{1}=\frac{r}{2||f_{0}||_{p}}\,,\, r_{2}=\frac{r}{2\lambda_{max}}$,所以数乘连续。
	\end{enumerate}
	\end{itemize}
\end{proof}

\begin{remark}
	\begin{itemize}
		\item 当$0\, <\, p\, <\, 1$时,$L_{p}(\Omega)$不是赋范空间。
		\item 当$1\, \leq p \leq \infty$,$L_{p}(\Omega)$是赋范空间。
	\end{itemize}
\end{remark}

\begin{theorem}\label{XQHthe070101}
	设$E$是拓扑向量空间,则“伸缩平移”是$E$上的同胚。
\end{theorem}

\begin{hint}
	连续映射的复合仍是连续映射。
\end{hint}

\begin{remark}
	注意这里的伸缩指的是$\lambda \neq 0 $的情形。
\end{remark}

\begin{definition}
	\begin{enumerate}
		\item 平衡的:若任取$|\lambda|\leq 1(\lambda\in \mathbb{K}) $,对$\forall x\in A$,有$\lambda x\in A $,那么称$A$是平衡的。
		\item 吸收的:对$\forall x\in E $,$\exists a>0 (constant)$,当取$|\lambda|\leq a(\lambda \in \mathbb{K})$时,有$\lambda \, x\in A$,那么称$A$是吸收的。
	\end{enumerate}
\end{definition}

\begin{remark}
	若$A$是平衡的,则$A$是对称的,即$-A=A$,且$0\in A$。
\end{remark}

\begin{theorem}
	若$E$是拓扑向量空间,$A\subset E$。则
	\begin{enumerate}
		\item 若$A$是向量子空间(凸集或平衡集),则$\bar{A}$也是向量子空间(凸集或者平衡集)。
		\item 若$A$是凸集,则$\mathring{A}$也是凸集。
		\item 若$A$是平衡集且$0\in \mathring{A} $,则$\mathring{A}$也是平衡集。
	\end{enumerate}
\end{theorem}

\begin{proof}
	\begin{enumerate}
		\item 
			\unfinished
			{\begin{itemize}
			\item $\forall x,y\in \bar{A} $,证明$x+y \in \bar{A}$
			\item $\forall x\in \bar{A} ,k\in \mathbb{K}  $,证明$k\, x\in \bar{A} $。\\
				$\forall V_{k\, x}\in \mathcal{N}(k\, x) $,考虑$\frac{1}{k} V_{k\, x } $。由定理\ref{XQHthe070101}可知$\frac{1}{k}V_{k\,x } \in \mathcal{N}( x ) $,由于$x\in \bar{A} $,所以$\exists \{x_{n}\} \subset A $ ,满足$x_{n} \rightarrow x \,\,(n\rightarrow\infty) $。则$\exists N>0$,使得$n > N$时,满足$x_{n} \in \frac{1}{k}V_{k\, x}$ 。则可证$k\, x_{n} \rightarrow kx (n\rightarrow\infty) $,所以$k\, x\in \bar{A} $。
			\end{itemize}
			}
		\finished
		{
			$\forall \lambda ,\beta \in \mathbb{K} $,考虑映射:
			\begin{equation*}
				{\large\text{$f_{\lambda,\beta}:\,\,\,\,$}}
				\begin{aligned}
					\bar{A}\times\bar{A}&\rightarrow E\\
					(x,y)&\rightarrow \lambda \, x+\beta \, y
				\end{aligned}
			\end{equation*}
		这里,$f_{\lambda,\beta}$是连续映射。要证$\bar{A}$是线性子空间,就是要证$f_{\lambda,\beta}(\bar{A}\times \bar{A})\subset \bar{A} $。这是因为$f_{\lambda,\beta}(\bar{A}\times\bar{A} ) = f_{\lambda,\beta }( \overline{A\times A} ) \subset \overline{ f_{\lambda,\beta}(A\times A) } \subset \bar{A}  $。\par
		这里补充证明一下:$f(\bar{B}) \subset \overline{f(B)} $(要求$f$连续)。$\forall f(x)\in f(\bar{B})  $(这里取$x\in \bar{B} $),$\forall V_{f_{x}}\in \mathcal{N}(f(x)) $,由$f$连续,则$f^{-1}(V_{f_{x}}) \in \mathcal{N}(x) $,又$x\in \bar{B} $,所以$f^{-1}(V_{f_{x}})\cap B\neq \emptyset $,所以$V_{f_{x}}\cap f(B) \neq \emptyset $,所以$f(x)\in \overline{f(B)} $。因此$f(\bar{B}) \subset \overline{f(B)} $。
		}
	  {课本}
	   对于凸集和平衡集的情况,也可以类似证明。
		\item $\forall 0<t<1 $,$t \mathring{A} + (1-t) \mathring{A} \subset A $,又$t \mathring{A} + (1-t) \mathring{A} $为开集,所以$t \mathring{A} + (1-t) \mathring{A}\subset \mathring{A} $。所以$\mathring{A} $为凸集。
		\item 要证$\forall |\lambda|\leq 1 $,$\lambda \mathring{A} \subset \mathring{A} $。对于$\lambda \neq 0 $,有$\lambda \mathring{A} \subset A $,且由定理\ref{XQHthe070101},$\lambda \mathring{A} $为开集,所以$\lambda \mathring{A} \subset \mathring{A} $。当$\lambda =0 $时,$\lambda \mathring{A}=0\in \mathring{A} $。所以$\mathring{A} $为平衡集。
	\end{enumerate}
\end{proof}

接下来用$\mathcal{N}(x)$表示$x$点的邻域系。
\begin{theorem}
	设$E$是拓扑向量空间。则
	\begin{enumerate}
		\item 任取$x\in E $,则$\mathcal{N}(x)=\mathcal{N}(0)+x $(即$V\in \mathcal{N}(x)\Leftrightarrow V-x\in \mathcal{N}(0) $)。
		\item 任取$V\in \mathcal{N}(0) $,存在$U\subset \mathcal{N}(0) $,使得$U+U\subset V $。
		\item 设$\lambda\in \mathbb{K} $且$\lambda\neq 0 $。则$V\in \mathcal{N}(0)\Leftrightarrow \lambda V\in \mathcal{N}(0) $。
		\item 任意$V\in \mathcal{N}(0) $是吸收的。
		\item 原点0有平衡的开(或闭)邻域基。
	\end{enumerate}
\end{theorem}
\begin{proof}
	\begin{enumerate}
		\item 任取$x\in E $,
		\begin{itemize}
			\item $"\Rightarrow"$,$\forall V \in \mathcal{N}(x) $,有$0\in V-x $,又由定理\ref{XQHthe070101},$V-x\in \mathcal{N}(0) $。
			\item $"\Leftarrow"$,$\forall U\in \mathcal{N}(0) $,有$U+x\in \mathcal{N}(x) $,取$U=V-x $,即得结论。
		\end{itemize}
	\item 考虑映射:
	\begin{equation*}
		\text{\large $f:$}
		\begin{aligned}
			E\, \times\, E\,&\rightarrow E\\
			(x,y)\,&\rightarrow x+y
		\end{aligned}
	\end{equation*}
	又拓扑向量空间的定义,这是连续映射,不妨设$V\in \mathcal{N}(0) $为开集。则$(0,0)\in f^{-1}(V) \subset E\, \times \, E $也为开集,所以$\exists U_{1},U_{2} \in \tau(E) $,且满足$U_{1},U_{2}\in \mathcal{N}(0) $,考虑$U= U_{1}\cap U_{2} $,即可验证满足定理结论。
	\item 由定理\ref{XQHthe070101}可得。
	\item $\forall V\in \mathcal{N}(0) $,$\forall x\in V $ ,考虑映射:
	\begin{equation*}
		\text{\large$\phi_{x}:$}\,\,
		\begin{aligned}
			\mathbb{K}\, & \rightarrow \, E\\
			k\, & \rightarrow kx.
		\end{aligned}
	\end{equation*}
	由拓扑向量空间的定义可得$\phi_{x} $为连续映射。$\phi_{x}^{-1}(V)$为$\mathbb{K}$的开集,且$0\in \phi_{x}^{-1}(V) $,所以$\exists \alpha >0 $,使得$B_{\mathbb{K}}(0,\alpha) \subset \phi_{x}^{-1}(V) $。从而结论成立。
	\item 
	\end{enumerate}
\end{proof}


\begin{theorem}\label{XQHthe070104}
	设$E$和$F$是两个拓扑向量空间,$u:\, E\, \rightarrow \, F $是线性映射。则以下命题等价:
	\begin{enumerate}
		\item $u$是连续映射。
		\item $u$在原点连续。
	\end{enumerate}
	若$F$还是赋范空间,则以上命题与如下命题等价:
	\begin{enumerate}
		\item[3.] $u$在原点的某邻域内有界。 
	\end{enumerate}
\end{theorem}

\begin{proof}
	
\end{proof}

\original
{定理\ref{XQHthe070104}意味着连续线性映射$u:E\, \rightarrow \, F $是一致连续的,即对$F$中原点的每个邻域$V$,存在$E$中原点的邻域$W$,使得当$x-y\in W $时,$u(x)-u(y) \in V $。
}
{P133}
\begin{remark}
	类比函数的一致连续理解。
\end{remark}

\begin{corollary}
	设$E$是数域$\mathcal{K}$上的拓扑向量空间,$u:E\, \rightarrow \, \mathbb{K} $是线性泛函。则定理\ref{XQHthe070104}的结论成立。
\end{corollary}


\subsection{半赋范空间}

\begin{definition}
	设$E$是数域$\mathbb{K}$上的向量空间,若$E$上的泛函$p: E\, \rightarrow \mathbb{R} $满足:
	\begin{itemize}
		\item 非负性
		\item 正齐性
		\item 三角不等式
	\end{itemize}
	则称$p$为$E$上的半范数。
\end{definition}

设$p$是$E$上的半范数。我们可定义相应的“半距离”为:
\begin{equation*}
	d_{p}(x,y)=p(x-y), \, \forall x,y\in E.
\end{equation*}
它显然满足距离的定义:
\begin{enumerate}
	\item $d_{p}(x,y)\geq 0  $,且$d_{p}(x,x)=0 $。
	\item $d_{p}(x,y)=d_{p}(y,x)$。
	\item $d_{p}(x,y) \leq d_{p}(x,z) +d_{p}(z,y) $。
\end{enumerate}
定义相应于半范数$p$的开球:
\begin{equation*}
	B_{p}(a,r)=\{x\in E: p(x-a)<r\},\, a\in E,r>0
\end{equation*}
称$B_{p}(a,r) $为一个开$p-$球,其中心为$a$,半径为$r$。\par
有了开球的概念,我们定义$E$上由半范数$p$诱导的拓扑$\tau_{p}$:若子集$A\subset E $可以表示成$E$中开$p-$球的并,则$A$为$E$中的开集。\original{容易验证所有这样的开集构成的集族的确是$E$上的拓扑}{P34},我们称该拓扑为$E$上的半范数$p$诱导的拓扑或$p-$拓扑。
\begin{proposition}
	验证,这样的开集构成的集族的确是$E$上的拓扑。
\end{proposition}
\begin{proof}
	\begin{itemize}
		\item $\emptyset ,E$可以写成开$p-$球的并。
		\item 任意并封闭。
		\item 有限交封闭。设$B_{1}=B_{p}(a_{1},r_{1}),B_{2}=B_{p}(a_{2},r_{2}) $。若$B_{1}\cap B_{2}=\emptyset$,则可以写成$p-$球的并。若$B_{1}\cap B_{2}\neq \emptyset$,则$\forall x\in B_{1}\cap B_{2} $,考虑$B_{x}=B_{p}(x,min\{r_{1}-p(a_{1},x),r_{2}-p(a_{2},x)\} ) $,可验证$B_{x}\subset B_{1}\cap B_{2} $。
	\end{itemize}
\end{proof}

\original{注意,$p-$拓扑是Hausdorff拓扑当且仅当$p$是一个范数。}{P134}正是因为单个半范数诱导的拓扑不一定是Hausdorff拓扑,我们往往从一族半范数出发,希望诱导出的拓扑是Hausdorff拓扑。
\begin{proposition}
	验证:$p-$拓扑是Hausdorff拓扑当且仅当$p$是一个范数。
\end{proposition}

\begin{proof}
	“$\Rightarrow$”,$\forall x\neq y$,因为$x-y\neq 0$,由$p-$拓扑是Hausdorff拓扑,所以$p(x-y)>0$,所以$p(x)=0$,则 $x=0$,$p$是范数。\par
	“$\Leftarrow$”,$\forall x\neq y $,因为$p$为范数,所以$p(x-y)\neq 0 $,考虑$B_{1}=B_{p}(x,\frac{p(x-y)}{2}),B_{2}=B_{p}(y,\frac{p(x-y)}{2})$,则可把$x,y$分开。
\end{proof}

设$(p_{i})_{i\in I} $是$E$上的一族半范数,$I$是指标集。
\original
{
	对任意有限子集$J\subset I $,令
	\begin{equation*}
		q_{J}=\max_{i\in J}\,p_{i}.
	\end{equation*}
	容易验证$q_{J}$也是$E$上的半范数。由此,我们定义如下半范数族诱导的拓扑。
}
{P134}

\begin{proposition}
	验证:$q_{J}$是$E$上的半范数。
\end{proposition}

\begin{proof}
	显然。
\end{proof}

\begin{definition}
	设$(p_{i})_{i\in I} $是$E$上的一族半范数。令$\tau$是由如下的子集组成的子集族:$O\in \tau $当且仅当
	\begin{equation*}
		O=\bigcup_{\alpha\in \Lambda}B_{q_{J_{\alpha}}}(x_{\alpha},r_{\alpha}),
	\end{equation*}
	其中$\Lambda $为某个指标集,每个$J_{\alpha} \subset I $是有限集,并且$x_{\alpha} \in E ,r_{\alpha}>0$。
\end{definition}

\begin{proposition}
	验证:$\tau$是$E$上的拓扑。
\end{proposition}

\begin{proof}
	\begin{enumerate}
		\item $\emptyset, E \in \tau $
		\item 任意并封闭。
		\item 有限交封闭。取$B_{1}=B_{J_{\alpha}}(x_{\alpha},r_{\alpha}), B_{2}=B_{J_{\beta}}(x_{\beta},r_{\beta}) $,若$B_{1}\cap B_{2}=\emptyset $,则结论成立。若$B_{1}\cap B_{2}\neq \emptyset $,$\forall x \in B_{1}\cap B_{2} $。考虑$B_{x}=B_{q'}(x,r_{x})$,这里$q'=\max\{q_{J_{\alpha}},q_{J_{\beta}}\} ,r_{x}=\min\{r_{\alpha}-p_{J_{\alpha}}(x,x_{\alpha}),r_{\beta}-p_{J_{\beta}}(x,x_{\beta}) \}$。
	\end{enumerate}
\end{proof}

\original
{
	如果$(p_{i})_{i\in I } $本身满足性质:任取$i,j\in I $,存在$k\in I $,使得
	\begin{equation*}
		p_{k}\geq \max\{p_{i},p_{j}\},
	\end{equation*}
	则我们不必定义新的$(q_{J})_{J\subset I} $。此时,我们称半范数族$(p_{i})_{i\in I} $是定向的。实际上,我们之所以由半范数族$(p_{i})_{i\in I} $来构造新的半范数族$(q_{J})_{J\subset I} $($J$为有限集),正是为了使新的半范数族称为一个定向集。\uline{若\text{$(p_{i})_{i\in I} $}是定向的,则用\text{$p_{i}-$}球定义的拓扑和用\text{$q_{J}-$}定义的拓扑是一致的。}
}
{P135}

\begin{proposition}
		验证:$若(p_{i})_{i}$是定向的,则用$p_{i}- $球定义的拓扑和用$q_{J}- $定义的拓扑是一致的。 
\end{proposition}

\begin{proof}
	用定义验证即可。
\end{proof}

\original
{
	由半范数族$(p_{i})_{i\in I} $诱导的拓扑是Hausdorff拓扑当且仅当$(p_{i})_{i\in I} $在$E $上是可分点的,即对$\forall x\in E $且$x\neq 0 $,$\exists i \in I $,使得$p_{i}(x)\neq 0 $。(等价地,若$\sup_{i\in I} p_{i}(x)=0$,则$x=0$)。
}
{P136}
\begin{proposition}
	\begin{itemize}
		\item 验证:半范数族诱导的拓扑是Hausdorff拓扑当且仅当$(p_{i})_{i\in I} $在$E$上是可分点的。
		\item 验证:此处可分点的定义是否等价于$\forall x,y\in E$,若$x\neq y $,则$\exists i\in I $,使得$p_{i}(x)\neq 0 $。
		\item 验证:此处可分点的定义等价于若$\sup_{i \in I} p_{i}(x) =0 $,则$x=0 $。
	\end{itemize}
\end{proposition}

\begin{proof}
	\begin{enumerate}
		\item “$\Rightarrow$”若$\exists x\neq  0$,使得$\sup_{i \in I}p_{i}(x)=0 $,则根据半范数族诱导的拓扑不能把$x,0$分开。\\
		“$\Leftarrow $”$\forall x,y\in E $,若$ x-y \neq 0$,因为$(p_{i})_{i\in I} $可分,所以$\exists i\in I $,满足$p_{i}(x,y)\neq 0 $,考虑$B_{1}=B_{p_{i}}(x,\frac{p_{i}(x,y)}{2}) ,B_{2}=B_{p_{i}}(y,\frac{p_{i}(x,y)}{2})$,可验证$B_{1}\cap B_{2} $。
	\end{enumerate}
\end{proof}

\begin{theorem}
	设$(p_{i})_{i\in I} $是向量空间$E$上的半赋范族。那么由$(p_{i})_{i\in I} $诱导的拓扑$\tau $与$E$的线性结构相容。因此,$E$成为一个拓扑向量空间,并且该拓扑是使每个$p_{i}(i\in I)$都连续并与线性结构相容的拓扑中最弱的拓扑。
\end{theorem}
\begin{proof}
	
\end{proof}

\begin{definition}
	\begin{itemize}
		\item 设$E$是向量空间,$(p_{i})_{i\in I} $是$E$上的一族半范数,则称$E$为半赋范空间,记为$(E,(p_{i})_{i\in I}) $。半赋范空间是拓扑向量空间。
		\item 若拓扑向量空间$E$的拓扑可由一族半范数诱导,则称$E$为可半赋范的。
	\end{itemize}
\end{definition}

\begin{theorem}\label{XQHthe070202}
	半赋范空间$E$的原点有一组由凸集组成的邻域基。
\end{theorem}

\begin{theorem}
	设$(E,(p_{i})_{i\in I}) $和$(F,(q_{j})_{j\in J}) $是两个半赋范空间,$ u: E\, \rightarrow \, F$是线性映射。那么$u$是连续的当且仅当对$J$的任意有限子集$J' $,存在$I$的有限子集$I'$及常数$C>0$,使得
	\begin{equation*}
		\max_{j\in J'} q_{j}(u(x))\leq C\max_{i\in I'} p_{i}(x),\quad \forall x\in E.
	\end{equation*}
	特别地,如果$F$是赋范空间,则$u$是连续线性映射当且仅当存在$I$的有限子集$I' $及常数$C>0 $,使得
	\begin{equation*}
		||u(x)||\leq C\max_{i\in I'}p_{i}(x),\quad \forall x\in E.
	\end{equation*}
\end{theorem}

\original
{
	设$(E,(p_{i})_{i\in I}) $和$(F,(q_{j})_{j\in J}) $是两个半赋范空间。令$K=I\times J $,并且对任意$k=(i,j) \in K $,定义
	\begin{equation*}
		r_{k}(x,y)=\max\{p_{i}(x),q_{j}(y)\},\quad \forall (x,y)\in E\times F.
	\end{equation*}
	那么$(r_{k})_{k\in K}$是$E\times F $上的一族半范数。可以证明,$(r_{k})_{k\in K} $是可分点的当且仅当$(p_{i})_{i\in I} $和$(q_{j})_{j\in J} $是可分点的。并且$E\times F$上由半范数族$(r_{k})_{k\in K} $诱导的拓扑与其上的乘积拓扑一致。
}
{P137半赋范空间的乘积空间}
\begin{proposition}
	\begin{itemize}
		\item $(r_{k})_{k\in K} $是可分点的当且仅当$(p_{i})_{i\in I} $和$(q_{j})_{j\in J} $都是可分点的。
		\item $E\times F$上由半范数族$(r_{k})_{k\in K} $诱导的拓扑与其上的乘积拓扑一致。
	\end{itemize}
\end{proposition}

\begin{proof}
	\begin{itemize}
		\item 按照可分点的定义验证即可。
		\item 记$\tau_{1}=\tau_{(r_{k})_{k\in K}} $,$\tau_{2}=\tau_{E\times F} $。\par
		先证$\tau_{1}\subset \tau_{2} $,任取$V\in \tau_{1} $,任取$(x,y)\in V $,$\exists B_{J_{k}}((x,y),r_{k})\subset V $。$J_{k} $对应指标集$J_{i},J_{j} $,考虑$B_{1}=B_{J_{i}}(x,r_{k}),B_{2}=B_{J_{j}}(y,r_{k})$,容易验证$B_{1}\times B_{2}\subset B_{J_{k}}((x,y),r_{k}) $,所以$V\in \tau_{2}$。\par
		再证$\tau_{2}\subset \tau_{1} $,$\forall U\in \tau_{2} $,以及$\forall (x,y)\in \tau_{2}$,$\exists B_{1}=B_{J_{i}}(x,r_{x}),B_{2}=B_{J_{j}}(y,r_{y}) $,满足$B_{1}\times B_{2}\subset V $,考虑指标集$J_{k}=J_{i}\times J_{j} $,以及$B_{0}=B_{J_{k}}((x,y),\min\{r_{x},r_{y}\}) $,容易验证$B_{0}\subset B_{1}\times B_{2} $,所以$U\in \tau_{1} $。所以$\tau_{2}\subset \tau_{1} $。
	\end{itemize}
\end{proof}	

\subsection{局部凸空间}
\hspace{2em}我们在上节看到可半赋范的拓扑向量空间的原点具有凸邻域基(定理\ref{XQHthe070202})。这一节证明其逆命题也成立。

\begin{proposition}
	一个拓扑向量空间$E$称为局部凸空间,若它的原点有一组由凸集组成的邻域基。
\end{proposition}

\begin{theorem}
	设$E$是一个局部凸空间,则其在原点有凸的平衡开(或闭)邻域基。
\end{theorem}

\original
{
	设$E$是拓扑向量空间,$\Omega$是$E$中原点处的凸平衡开邻域。我们称泛函$p_{\Omega}:E\, \rightarrow \mathbb{R} $,
	\begin{equation*}
		P_{\Omega}(x)=\inf\{\lambda >0:\frac{x}{\lambda}\in \Omega\}
	\end{equation*}
	为相应于$\Omega $的$Minkowski$泛函。
}
{P139 Minkowski泛函}
\begin{remark}
	由平衡的定义,若$\lambda \geq 1 $,则$\frac{x}{\lambda} \in \Omega $,所以$ p_{\Omega}(x)\leq 1$。
\end{remark}

\original
{
	我们给出几个$Minkowski$泛函的性质:
	\begin{enumerate}
		\item 设
		\begin{equation*}
			I(x)=\{\lambda>0:\frac{x}{\lambda}\in \Omega\}.
		\end{equation*}
	若$\lambda\in I(x) $且$\mu >\lambda $,则$\mu \in I(x) $。因为$\Omega $是平衡的,所以有
	\begin{equation*}
		\frac{x}{\mu}=\frac{\lambda}{\mu}\, \frac{x}{\lambda}\in \Omega.
	\end{equation*}
	故$I(x) $是以$p_{\Omega}(x) $为左端点的半直线。
	
	\item $\forall \lambda_{0}\in I(x) $,$\exists \varepsilon>0 $,使得
	\begin{equation*}
		(\lambda_{0}-\varepsilon,\lambda_{0}+\varepsilon)\subset I(x).
	\end{equation*}
	因为\text{$\lambda\,\rightarrow \,\frac{x}{\lambda} $}为连续映射,且$\Omega $为开集,所以$\exists \varepsilon>0 $,满足$(\lambda_{0}-\varepsilon,\lambda_{0}+\varepsilon ) \subset I(x)$,这说明$I(x) $为开集,可表示为$I(x)=(P_{\Omega}(x),\infty) $。因此$p_{\Omega}<1 $等价于$1\in I(x) $,即$x\in \Omega $。
	\end{enumerate}
}
{P139}

\begin{theorem}
	设$E$是一个拓扑向量空间,设$\Omega,\Omega_{1},\Omega_{2} $都是$E $中原点的凸平衡开邻域。则有
	\begin{enumerate}
		\item Minkowski泛函$p_{\Omega}:E\, \rightarrow \mathbb{R} $是$E$上的半范数,并有
		\begin{equation*}
			\Omega=\{x\in E:p_{\Omega}(x)<1\}.
		\end{equation*}
		
		\item 若$\Omega_{1}\subset \Omega_{2} $,则$I_{\Omega_{1}}(x)\subset I_{\Omega_{2}}(x) $,从而$p_{\Omega_{2}}\leq p_{\Omega_{1}} $。
		
		\item 存在$\Omega_{3}=\Omega_{1}\cap\Omega_{2} $,使得$p_{\Omega_{3}}\geq \max \{p_{\Omega_{1}},p_{\Omega_{2}}\} $。
	\end{enumerate}
\end{theorem}

\begin{proof}
	\begin{enumerate}
		\item \begin{itemize}
			\item 非负性,显然。
			\item 正齐性。\par
			要证$p_{\Omega}(\mu x) =\inf_{\lambda}\{\lambda>0:\frac{\mu x}{\lambda}\in \Omega \}=|\mu|p_{\Omega}(x)(=|\mu|\inf_{\lambda}\{\lambda>0:\frac{x}{\lambda}\in \Omega\})$。$\forall \lambda \in \{\lambda>0: \frac{\mu x}{\lambda}\in \Omega\} $,则由于$\Omega $是平衡的,所以$\frac{x}{\frac{\lambda}{|\mu|}} \in \Omega$,所以$\frac{\lambda}{|\mu|} \geq p_{\Omega}(x)$,所以$p_{\Omega}(\mu x)\geq |\mu|p_{\Omega}(x) $。\par
			$\forall \lambda\in \{\lambda>0:\frac{x}{\lambda}\in \Omega\} $,由于$\Omega $是平衡的,所以$\frac{\mu x}{|\mu|\lambda}\in \Omega $,所以$|\mu|\lambda\geq p_{\Omega}(\mu x) $,所以$|\mu|p_{\Omega}(x)\geq p_{\Omega}(\mu x) $。
			\item 三角不等式。$\forall x,y\in E $,若$\lambda_{x}>0,\lambda_{y}>0$,使得$\frac{x}{\lambda_{x}}\in \Omega,\frac{y}{\lambda_{y}}\in \Omega $,考虑
			\begin{equation*}
				\frac{x+y}{\lambda_{x}+\lambda_{y}}=\frac{\lambda_{x}}{\lambda_{x}+\lambda_{y}}\frac{x}{\lambda_{x}}+\frac{\lambda_{y}}{\lambda_{x}+\lambda_{y}}\frac{y}{\lambda_{y}}\in \Omega
			\end{equation*}
		所以$p_{\Omega}(x+y)\leq p_{\Omega}(x)+p_{\Omega}(y) $。
		\end{itemize}
	\item 显然。
	\item 可以验证$\Omega_{1}\cap\Omega_{2} $为平衡开凸集。$p_{\Omega_{3}}\geq\max\{p_{\Omega_{1}},p_{\Omega_{2}}\} $容易验证。
	\end{enumerate}
\end{proof}










