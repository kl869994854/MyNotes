\section*{引言}

\subsection*{一笔画问题和七桥问题}

\subsection*{地图着色问题}

\subsection*{Euler多面体定理}

\original
{
    拓扑性质体现的是图形整体结构上的特性,可以随意把图形作变形(如挤压、拉伸或扭曲等等),只要不把它撕裂,不发生粘连,从而不破坏其整体结构,拓扑性质将保持不变。把上述变形称为图形的“拓扑变换”,那么\uline{拓扑性质就是几何图形在作拓扑变换时保持不变的性质。拓扑变换可用几何与映射的语言给出确切的描述。把图形 $M$变形为 $M'$,就是给出 $M$到 $M'$(都看作点集)的一个一一对应(因而不出现重叠现象,也不产生新点) $f:M\rightarrow M'$,并且 $f$连续(表示不撕裂), $f^{-1}:M'\rightarrow M$也连续(表示不粘连)。}
}
{P5}

\begin{proposition}
    \begin{enumerate}
        \item 如何定义“重叠现象”,“产生新点”,“撕裂”,“粘连”?
        \item 为什么要求映射一一对应就能保证不出现重叠现象也不产生新点,要求$f$连续就保证不撕裂,$f^{-1}$连续就保证不粘连。
    \end{enumerate}
\end{proposition}

\begin{proof}
    \begin{enumerate}
        \item 
        \item $f^{-1}$连续即从$M'$映到$M$不撕裂。
    \end{enumerate}
\end{proof}

\original
{
    简单来说:从图形$M$到$M'$的一个一一对应$f$,如果$f$与$f^{-1}$都是连续的,就称$f$为从$M$到$M'$的一个拓扑变换,并称$M$与$M'$是同胚的。于是,拓扑性质也就是同胚的图形共有的性质。
}
{P5}

\begin{enumerate}
    \item \textbf{Brouwer不动点定理}
    \item \textbf{Jordan曲线定理}
\end{enumerate}