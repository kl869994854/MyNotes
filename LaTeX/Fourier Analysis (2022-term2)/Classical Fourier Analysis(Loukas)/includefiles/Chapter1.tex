\chapter{\texorpdfstring{$L^p$}. Spaces and Interpolation}

\section{\texorpdfstring{$L^p$}. and Weak \texorpdfstring{$L^p$}.}

\subsection{The Distribution Function}

\begin{proposition*}[1.1.3]
obvious.
\end{proposition*}

\begin{proposition*}[1.1.4]
\begin{equation*}
\norm{f}_{L^p}^p
=
p
\int_{0}^{\infty}
\alpha^{p-1}
\dd _f (\alpha)
\dd \alpha       
\end{equation*} 
\end{proposition*}

\begin{proof}
    P5
\end{proof}

\noindent Define $L^{\infty,\infty}=L^{\infty}$.
\\
\noindent $L^{p,\infty}$ is a {\it quasi-normed} linear space.
\\
\noindent $L^p\subset L^{p,\infty}$ is strict.

\begin{example*}
    Consider $h(x)=|x|^{-\frac{n}{p}}$ .It is obvious that $h(x)\not\in L^p$.
    \par
    But 
    \begin{equation*}
        \alpha \dd_h^{
            \frac{1}{p}
        }(\alpha)
        =\alpha\mu
        \{
        |x|^{-\frac{n}{p}}>\alpha
        \}^{\frac{1}{p}}
        =
        \alpha\mu\{
            |x|<\alpha^{-\frac{p}{n}}
        \}^{\frac{1}{p}}
        =\nu_n^{\frac{1}{p}}
    \end{equation*}
    $\mu$ is lebesgue measure and $\nu_n$ is the lebesgue measure of $B_1$ . The last equality is by the fact that $\mu\{B_n(R)\} \propto R^n$ . 
\end{example*}

\subsection{Convergence in Measure}

\begin{proposition*}[1.1.9]
    $f_n\xrightarrow{\norm{\cdot}_{L^p}}f\,\,\Rightarrow \,\, f_n\xrightarrow{\norm{\cdot}_{L^{p,\infty}}}f\,\,
    \Rightarrow \,\,
    f_n\xrightarrow{\mu} f
    $
\end{proposition*}
There is no general converse of statement of {\it Proposition 1.1.9}.
\begin{example*}
    Consider $f_{k,j}=k^{\frac{1}{p}}\chi_{[\frac{j}{k},\frac{j-1}{k}]}$ where $p>0$ , $k\leq 1$ , $1\leq j \leq k$ . Then show that $\{f_n\}=\{f_{1,1},f_{2,1},f_{2,2},f_{3,1},\cdots \}$ is converge to $0$ in measure , but not converge to $0$ in $\norm{\cdot}_{L^{p,\infty}}$ .
    \begin{itemize}
    \item {}
    $\{
    |f_{k,j}|>0    
    \}<\frac{1}{k}$ , so $\{f_n\}\xrightarrow{\mu} 0$.

    \item {}
    $
    \norm{f_{k,j}}_{L^{p,\infty}}
    =
    \sup_{\alpha>0}
    \alpha
    \dd _{f_{k,j}}^{\frac{1}{p}}(\alpha)
    \geq
    \frac{(k-\frac{1}{k})^{\frac{1}{p}}}
    {k^{\frac{1}{p}}}
    $ is valid for $\forall k\geq 1$ . In fact , the right of the ineuqality is just the case $\alpha=\left(k-\frac{1}{k}\right)^{\frac{1}{p}}< k^{\frac{1}{p}}$ . While 
    $$
    \lim_{k\rightarrow \infty}\left(
        k-\frac{1}{k}
    \right)^{\frac{1}{p}}
    =1
    $$
    \myend
    \end{itemize} 
\end{example*}

\begin{theorem*}[1.1.11]
$$
f_n\xrightarrow{\mu}f
\,\,
\Rightarrow
\,\,
f_n\xrightarrow{\mu-a.e.}f
$$
\end{theorem*}
\begin{proof}
    P8
\end{proof}
\begin{remark*}
    Show that the null set 
    $$
    \bigcap_{m=1}^{\infty}
    \bigcup_{k=m}^{\infty}
    A_k
    $$
    contains the set of all $x\in X$ for which $f_{n_k}(x)$ does not converge to $f(x)$.
\end{remark*}
\begin{proof}
    By definition of $f_{n_k}(x)\not\rightarrow f(x) $ in $x$ : $\exists\, \epsilon_0>0$ (assume $2^{-(k_0-1)}\geq \epsilon_0\geq 2^{-k_0}$) , $\forall \, N>0$ , $\exists \,k>N$ , have :
    $$
    |f_{n_k}(x)-f(x)|\geq \epsilon_0\geq 2^{-(k_0-1)}
    $$
    then for $\forall \, m\geq 1$ , let $N=\max\{m,k_0-1\}$ , then $\exists \, k>N$
    $$
    |f_{n_k}-f(x)|
    \geq
    2^{-k_0-1}
    >2^{-k}
    $$
    so $x\in A_k$ ($k>N>m$).
     So $x\in \bigcap_{m=1}^{\infty}
     \bigcup_{k=m}^{\infty}
     A_k$ .
\end{proof}

\subsection{A First Glimpse at Interpolation}

\question

\begin{enumerate}
    \item (P9) Show that $f\in L^r(X,\mu)$ , $\forall \, p<r<q$ , if $f\in L^p(X,\mu)\cap L^q(X,\mu)$.
    \item (P11) Show that $f(x)=|x|^{-n-\alpha}\chi_{|x|\leq 1}$  is not integrable over any open set in $\mathbb{R}^n$ containing the origin.
\end{enumerate}

\solution

\begin{enumerate}
    \item H{\'o}lder Ineuqality : decompose $|f|=|f|^{1-\theta}|f|^{\theta}$.
    \item tranform the coordinate in ball's coordinates.
\end{enumerate}

\section{Convolution and Approximate Identities}

\subsection{Examples of Topological Groups}

\subsection{Convolutiln}

\question

Why $\int_G\,
f(tx) 
\,\dd \lambda(x)
=
\int_G\,
f(x)
\,\dd \lambda(x)
$
?
\par
This pushes me to review the definition of Lebesgue integration.
\begin{example*}
\begin{equation*}
\int_{0}^1\,
x+t
\,\dd x
=
\int_{0+t}^{1+t}\,
y
\, \dd y
\end{equation*}
\end{example*}

\solution
Note that $G$ is a locally compact group! $G=tG$!

\begin{definition*}[1.2.6 Convolution]
    Let $f,g$ be in $L^1(G)$ . Define the convoluiton:
    \begin{equation*}
        f\star g(x)
        =
        \int_G\,
        f(y)
        g(y^{-1}x)
        \,\dd \lambda (y)
    \end{equation*} 
\end{definition*}

Show that the definition is well-defined . In fact , $|f\star g| \in L^1(G)$ ,then implies that $f\star g(x)$ is finite a.e. (otherwise , if $\left|\{f\star g(x)=\infty\}\right|>0$ , contracdiciton!).
\begin{proof}
\begin{align*}
\norm{|f\star g|}_{L^1(G)}
&=
\int_G\,
\left|
f\star g(x)
\right|
\,\dd x  \\
&=
\int_G\,
\left|
\int_G\,
f(y) g(y^{-1}x)
\,\dd y
\right|
\,\dd x  \\
&\leq
\int_G\,
\int_G\,
|f(y)|
|g(y^{-1}x)|
\,\dd y
\,\dd x \quad (\text{Fubini Theroem})\\
&=\norm{f}_{L^1(G)}\, \norm{g}_{L^1(G)}
<\infty
\end{align*}
\end{proof}


\begin{example*}[1.2.8]
    $f(x)=1$ when $-1\leq x\leq 1$ , calculate $f\star f$ , $f\star f\star f$ ,and observe that it gets smoother with more $\star$ .
\end{example*}
\begin{remark*}
    And let $g=\chi_{B_1(0)}$ and calculate $g\star g$ .
\end{remark*}

\begin{proof}
    

\end{proof}

Finally , show that $L^1(G)$ is a Banach algebra under the convolution product.

\begin{proof}
    
\end{proof}

\subsection*{Lebesgue Integration}
\noindent (From Stein <real analysis> chapter2 section1)
\\
Every step can be roughly divided into 4 parts.
\begin{itemize}
    \item definition of integral and integrable .
    \item show the definition is well-defined.
    \item property of lebesgue integration.
    \item convergence theorem.
\end{itemize}
Especially in {\bfseries Step2} , connection between lebesgue integration and Reimann integration has been shown.
\subsubsection*{Step1 simple funtions}
If $f$ is a simple function. And $f(x)=\sum\limits_{k=1}^{n}\alpha_k \chi_{E_k}(x)$ .
Then
\begin{equation*}
    \int\, f(x) \,\dd  x=\sum\limits_{k=1}^{n}\alpha_k |E_k|
\end{equation*}

\subsubsection*{Step2 Bounded funcitons supported by bounded sets}

\begin{equation*}
    \int \, f(x) \, \dd x
    =
    \lim_{n\rightarrow\infty}
    \int \,
    \phi_n(x)
    \,\dd x 
\end{equation*}

\subsubsection*{Step3 positive functions}
\subsubsection*{Step4 general functions}

\begin{theorem*}[1.13 lebesgue dominated convergence theorem]

\end{theorem*}


\subsubsection*{Complex-valued functions}
Given a complex-valued function $f(x)=u(x)+\imath v(x)$ , where $u(x)$ , $v(x)$ are real-valued function. Then $f$ is {\bfseries Lebesgue integrable} $\Leftrightarrow$ $|f(x)|=(|u(x)|^2+|v(x)|^2)^{\frac{1}{2}}$ is {\bfseries Lebesgue integrable} . And $\int f= \int u +\imath \int v$ .
\par
An observation is that $f(x)$ is lebesgue integrable $\Leftrightarrow$ $u(x)$ , $v(x)$ is lebesgue integrable .
\begin{proof}

\end{proof}

Definition of Banach algebra:

\subsection{Basic Convolution Inequalities}

Three Inequalities in this section:
\begin{itemize}
    \item Minkowski's ineuqality
    \item Young's ineuqality
    \item Young's inequality for weak type spaces(condition is weaker and conclution is weaker than Young's inequality)
\end{itemize}

\subsubsection{Approximate Identities}


\section{Interpolation}

\section{Lorentz Spaces}
