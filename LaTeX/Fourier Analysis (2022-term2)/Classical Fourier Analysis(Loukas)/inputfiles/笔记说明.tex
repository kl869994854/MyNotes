%姚慕生,笔记的说明信息。
\usepackage{hyperref}
\usepackage{amsmath}
\usepackage{amsfonts}
\usepackage{amsthm}
\usepackage{physics}
\usepackage{amssymb}
\usepackage{color,xcolor}
%\usepackage{ntheorem}


\hypersetup{
colorlinks=true,
linkcolor=black
}

{



\theoremstyle{definition}  


\newtheorem*{example*}{Example}
\newtheorem{example}{Example}

\newtheorem*{proposition*}{\bfseries Proposition}
\newtheorem{proposition}{\bfseries Proposition}



\newtheorem*{definition*}{\bfseries Definition}
\newtheorem{definition}{\bfseries Definition}

\newtheorem{theorem}{\bfseries Theorem}
\newtheorem*{theorem*}{\bfseries Theorem}

\newtheorem{lemma}{\bfseries Lemma}
\newtheorem*{lemma*}{\bfseries Lemma}

}

{
\theoremstyle{remark}
\newtheorem*{remark*}{\bfseries Remark}
}


\newcommand{\question}[0]{\noindent\textcolor{red}{\large\bfseries Question:\vspace{1ex}}}

\newcommand{\answer}[1]{\noindent\textcolor{cyan}{\large\bfseries Answer:}{ \,\,#1}\hfill \mbox{}  \hfill $\blacksquare$}

\newcommand{\solution}[0]{\noindent\textcolor{cyan}{\large\bfseries Solution:}}

\newcommand{\myend}[0]{\hfill \mbox{}\hfill $\blacksquare$}

\author{MMW}
\date{Starting date:2022/4/18}
\title{Classic Fourier Analysis\footnote{by Loukas Grafakos}}

\endinput