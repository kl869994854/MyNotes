\section{The Direct Methos in the Calculus of Variations}

\subsection{Lower Semi-continuity}

\begin{theorem*}[1.1]\label{C1S1_1.1}
    
\end{theorem*}
\begin{remark*}
    In the proof of the case of sequentially compact, how to get the existence of u $\in \bigcap_{\alpha>\alpha_0}K_\alpha$:
    $\forall \, n\in N$, beacause $E(u_m)\rightarrow \alpha_0$, so $\exists \, M$, when $m>M$, $u_m\in K_{\alpha+\frac{1}{n}}$, and for $K_{\alpha_0+\frac{1}{n}}$ is sequentially compact, there is a subsequence of \{$u_m$\} denoted by $\{u_m^{(n)}\}$ that converge to $u^{(n)}\in K_{\alpha_{0}+\frac{1}{n}}$. Similarily, there is a subsequence of $(u_m^{(n+1)})$ that converge to $u^{(n+1)}\in K_{\alpha_0+\frac{1}{n+1}}$. And M is Huausdorff space, so $u_m^{(n+1)}=u_m^{(n)}$. So take $u_m^{(n)}$ as $u$. Then we can prove that $u\in \bigcap_{\alpha>\alpha_0}K_{\alpha}$.
\end{remark*}

\subsection{}

\subsection{}

\subsection{The Concentration-Compactness Principle}


\begin{lemma*}[4.3 Concentration-Compactness Lemma 1]\label{C1S4_4.3}
    
\end{lemma*}

\begin{lemma*}[4.8 Concentration-Compactness Lemma 2]\label{C1S4_4.8}
    
\end{lemma*}

\begin{theorem*}[4.9]\label{C1S4_4.9}
    
\end{theorem*}

\begin{remark*}
I divide the proof into three parts (naturally):
\begin{enumerate}
    \item {}
    Using the conditions that $\{u_m\}$ is the mininizing sequence of $S$ , $\norm{u_m}_{L^q}=1$ and the definiton of $Q(r)$, author constructs $\{vm\}$:
    \begin{equation*}
      v_m=
      \tilde{R}_m^{-n/q}
      u_m
      \left(
          \dfrac{x-\tilde{x}_m}
          {\tilde{R}_m}
      \right)      
    \end{equation*}
    and $\{v_m\}$ satisfy:
    \begin{enumerate}
    \item 
    $\{v_m\}$ is mininizing sequence of $S$.
    
    \item {}
    $\norm{v_m}_{L^q}=1$.

    \item {}
    $Q(1)=\sup_{x\in\mathbb{R}^n}
    \int_{B_1(x)}\,|v_m|^q\, \dd x
    =
    \int_{B_1(0)}\,|v_m|^q\,\dd x
    =
    \dfrac{1}{2}$
    \end{enumerate}
and constructs :
\begin{equation*}
    \mu_m=|\nabla^k v_m|^p\dd x\, , \, \nu_m=|v_m|^q\dd x
\end{equation*}

\item {}
Using \hyperlink{C1S4_4.3}{lemma 4.3} to $\{v_m\}$ , and discuss to show that only case (1.Compactness) is valid.

\item {}
Then use \hyperlink{C1S4_4.9}{lemma 4.8} to $\mu_m,\nu_m$.

\end{enumerate}
\end{remark*}

\question

\begin{enumerate}
\item {\bfseries done}%1
existence of $\tilde{R}_m$ , $\tilde{x}_m$ ?
\item {}%2
How to prove $v_m\rightharpoondown v$ in $D^{k,p}(\mathbb{R}^n)$ ?
\item {}%3
How to prove 
\begin{equation*}
    \norm{(\nabla^k v_m)\phi_m}_{L^p(\mathbb{R}^n)}
    \geq
    \norm{\nabla^k(v_m\phi_m)}_{L^p(\mathbb{R}^n)}
    -
    C\sum_{l<k}
    \norm{\nabla^l v_m\nabla{k-l}\phi_m}_{L^p(\mathbb{R}^n)}
\end{equation*}
by Minkowski's inequality ? And what it means by $\norm{\nabla^k(v_m\phi_m)}_{L^p(\mathbb{R}^n)}$ ?

\item {}%4
Young's inequality.

\item {}%5
interpolation.
\end{enumerate}

\solution

\begin{enumerate}
    \item {}
    If $Q(1)=\sup_{x\in\mathbb{R}^n}
    \int_{B_1(x)}\,
    |v_m|^q
    \,\dd x
    =\frac{1}{2}$
    has been proved , using the condition that $v_m\in C_0^{\infty}(\mathbb{R}^n)$ and doing translation to $x$ can get the conclution.
    \\
    See that :
    \begin{equation*}
    \begin{aligned}
    Q(1)
    &=
    \sup_{x\in\mathbb{R}^n}
    \int_{B_1(x)}\,
    |v_m(y)|^q
    \,\dd y
    \\
    &=
    \sup_{x\in\mathbb{R}^n}
    \int_{B_1(x)}\,
    |
    \tilde{R}_m^{-n/q}
    u_m
    \left(
        \dfrac{y-\tilde{x}_m}
        {\tilde{R}_m}
    \right)
    |^q
    \,\dd y
    \\
    &=
    \sup_{x\in\mathbb{R}^n}
    \int_{
        B_{\frac{1}{\tilde{R}_m}}
        (\frac{x_-\tilde{x}_m}{\tilde{R}_m})
    }\,
        |u_m(t)|^q
    \,\dd t
    {}
    \end{aligned}
    \end{equation*}
    Although centre of $B_{
        \frac{1}{\tilde{R}_m}
    }
    \left(
        \frac{x-\tilde{x}}{\tilde{R}_m}
    \right)
    $ is changed with $\tilde{R}_m$ , $x\in \mathbb{R}^n$ still make the centre go over whole $R^n$ .
    \\
    Next , show that $\sup_{y\in \mathbb{R}^n}
    \int_{B_{\frac{1}{\tilde{R}_m}}(y)}\,
    |u_m(t)|^q
    \,\dd t
    $ is continious in $\tilde{R}_m$ .
    \\
    Show that $\forall \, y\in R^n$ ,
    \begin{equation*}
        \left|
        \int_{B_{
        r+\Delta r
        }(y)
        }\,
        |u_m(t)|^q
        \,\dd t
        -
        \int_{
            B_{r}(y)
        }\,
        |u_m(t)|^q
        \,\dd t
        \right|
        \leq C\Delta r
    \end{equation*}
    $C$ is independent of $y$ . This can implies $\sup\cdots$ is continious in $\tilde{R}_m$.
    \begin{equation*}
    \begin{aligned}
    &    \left|
    \int_{B_{
    r+\Delta r
    }(y)
    }\,
    |u_m(t)|^q
    \,\dd t
    -
    \int_{
        B_{r}(y)
    }\,
    |u_m(t)|^q
    \,\dd t
    \right|
    (\text{let $t=\frac{r+\Delta r}{r} (x-y)+y$})
    \\
    &\leq
    \left|
    \int_{
        B_r(y)
    }\,
    |
    u_m
    (\frac{r+\Delta r}{r} (x-y)+y)
    |^q
    |
    \frac{r+\Delta r}{r}
    |^n
    \,\dd x
    -
    \int_{B_r(y)}\,
    |u_m(x)|^q
    \,\dd x
    \right|
    \\
    &\leq
    C \Delta r
    \end{aligned}
    \end{equation*}
    and claim that $\exists [a,b]$ , so that 
    $$
        \sup_{y\in\mathbb{R}^n}
        \int_{
        B_{
            \frac{1}{a}
            }(y)
            }\,
        |u_m(t)|^q
        \,\dd t
        <\frac{1}{2}
    $$
    and 
    $$
    \sup_{y\in\mathbb{R}^n}\int_{
        B_{
            \frac{1}{b}
        }(y)}
        |u_m(t)|^q\dd t
        >\frac{1}{2}
    $$
    beacause $u\in C_0^{\infty}$ , so have  $supp\,u_m(x)=\Omega$ and $|u_m|^q\leq D$ .
    \\
    Then let:
    \begin{align*}
        b&=\frac{1}{diag{\Omega}}\\
        a&=4D
    \end{align*}
    \item {}%2
    Show that $D^{k,p}$ is seperable.
    \item {}%3
    $
    |\nabla^k (v_m \phi_m)|
    =
    \sum\limits_{l\leq k}
     |\nabla ^l v_m| 
     \,
     |\nabla ^{k-l} \phi_m|
     $?(maybe wrong!)
    
    \item {}%4 
    When estimate $\beta_m$ , take $p=2$ for example . Then apply Young's inequality to the crossing terms.
    \item {}%5 
    Adams <Sobolev Spaces> Second edition Theorem 5.12.
\end{enumerate}

