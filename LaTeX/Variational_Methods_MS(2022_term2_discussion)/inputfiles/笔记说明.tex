%姚慕生,笔记的说明信息。
\usepackage{hyperref}
\usepackage{amsmath}
\usepackage{amsfonts}
\usepackage{amsthm}
\usepackage{physics}
\usepackage{amssymb}
\usepackage{color,xcolor}
%\usepackage{ntheorem}


\hypersetup{
colorlinks=true,
linkcolor=black
}

\newtheorem*{theorem*}{Theorem}

\newtheorem*{lemma*}{Lemma}

{
\theoremstyle{remark}
\newtheorem*{remark*}{\bfseries Remark}
}


\newcommand{\question}[0]{\noindent\textcolor{red}{\large\bfseries Question:\vspace{1ex}}}

\newcommand{\answer}[1]{\noindent\textcolor{cyan}{\large\bfseries Answer:}{ \,\,#1}\hfill \mbox{}  \hfill $\blacksquare$}

\newcommand{\solution}[0]{\noindent\textcolor{cyan}{\large\bfseries Solution:}}

\newcommand{\myend}[0]{\hfill \mbox{}\hfill $\blacksquare$}

\author{MMW}
\date{Starting date:2022/3/5}
\title{Variational Methods\footnote{by Michael Struwe}  Notes\footnote{2022 first-grade spring. discussion class with Liao Xin ,Fan Yunlu}}

\endinput