\section{2022/2/15}

\subsection{equations}
equations that are going to be learned in this course:
\begin{enumerate}
    \item Laplace: $\Delta u=0$.
    \item Heat: $u_t-\Delta u=0$.
    \item Schordinger: $\imath u_t -\Delta u=0$.
    \item Wave: $u_{tt} -\Delta u=0$.
    \item Hamilton-Jacobi: $u_t + H(D_x u)=0$.     
\end{enumerate}

and there are other equations:



\subsection{classification}

then classification of PDEs

\begin{enumerate}
    \item in order : first,second,third,...-order.
    \item linear,semilinear,quasilinear,fully nonlinear.
    \item in shape:
        \begin{itemize}
            \item elliptic equation(from Wiki):
            \url{https://encyclopedia.thefreedictionary.com/Elliptic+operator}
            \item parabolic equation
            \item hyperbolic equation
            \item mixed
        \end{itemize}
\end{enumerate}

\subsection{two tools}

\subsubsection{Integration by parts}

\subsubsection{Maximum principle}

\subsection{a priori estimate}

A example : continuity method.
\begin{equation*}
    \begin{aligned}
        \det{g_{ij}+u^t_{i\bar{j}}} &= e^{tf}\det{g_{i\bar{j}}},\quad t\in [0,1]\\
        t&=0:u^t=0\\
        I&=\{t:\exists sol u^t\in C^2\}
        \begin{cases}
            open\\
            closed(Yau , Fields)
        \end{cases}
    \end{aligned}
\end{equation*}

\subsection{Laplace's equation}
Laplace's equation:
\begin{equation*}
    \Delta u=0,\quad x\in \mathbb{R}^n 
\end{equation*}



then do a priori estimate to Laplace equation.\par

mean value property.\par

harmonic function: