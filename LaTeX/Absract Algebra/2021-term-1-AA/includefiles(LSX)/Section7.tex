\section{域论*}
\subsection{Galois基本理论}
\original
{
	本节中我们介绍著名的Galois理论的基本定理。这是对整个数学发展起重要推动作用的理论之一。
	\par
	我们仍局限于讨论特征为0的域的情形。可以完全设想我们就是在复数域$\mathbb{C}$中进行的。
	\par
	设$K/F$是$F$上分裂域。这里把下面将用到的域论中或群论中的一些结果重温一下:
	\begin{enumerate}
		\item $F$上分裂域和$F$上正规扩域是一回事;
		\item $F$上分裂域$K=F(\theta)=F(\theta_{i})=F(\theta_{1},\theta_{2},\cdots,\theta_{n})$,其中$\{\theta=\theta_{1},\theta_{2},\cdots,\theta_{n}\}$是$\theta$的$F-$最小多项式$p(x)$的全部根;
		\item $F$上分裂域$K=f(\theta)$,而$\theta$的$F-$最小多项式$p(x)$的次数为$n$,则有
		\begin{equation*}
			n=[K:F]=|Gal(K/F)|;
		\end{equation*}
		\item $F\subseteq L \subseteq K,F,L,K$是域,则有$[K:F]=[K:L][L:F]$;
		\item $\{e\}\subseteq H\subseteq G$,$\{e\}$(单位元组成的群)、H、G是有限群,则有
		\begin{equation*}
			|G|=[G:\{e\}]=[G:H][H:\{e\}]=[G:H]\cdot|G|;
		\end{equation*}
	
		\item 若G是有限群,则对任意$g\in G$有$gG=Gg=G$。
	\end{enumerate}
	取定F上分裂域$K=F(\theta)$,设$\theta$的一个$F-$极小多项式为$p(x)$,$G=Gal(K/F)$。令
	\begin{equation*}
		\begin{aligned}
			&\mathbb{K}=\{K/F\text{的中间域}\,:\, F\subseteq L\subseteq K \} ,\\
			&\mathbb{G}=\{G=Gal(K/F)\text{的一切子群}H\, :\, G\supseteq H\supseteq\{e\}\}.
		\end{aligned}
	\end{equation*}
}
{P175}
\begin{proposition}
	\begin{itemize}
		\item 验证:$F$上分裂域$K=F(\theta)=F(\theta_{i})=F(\theta_{1},\theta_{2},\cdots,\theta_{n})$,其中$\{\theta=\theta_{1},\theta_{2},\cdots,\theta_{n}\}$是$\theta$的$F-$最小多项式$p(x)$的全部根。
		
	\end{itemize}
\end{proposition}

\begin{proof}
	\begin{itemize}
		\item \unfinished{(反例,$f(x)=x^{3}-2$,$F=\mathbb{Q}$)已知$\theta$是$F$上的代数元,$f(x)\in F[x]$是$\theta$的极小多项式,若$K$是$f(x)$的分裂域,那么有$K=F(\theta)=F(\theta_{i})=F(\theta_{1},\cdots,\theta_{n})$。}
		\par
		若任给一多项式$f(x)\in F[x]$的分裂域$K$,那么$K$是单扩域吗?
	\end{itemize}
\end{proof}









\subsection{一个例子}

\begin{example}
	设K是多项式$f(x)=(x^{2}+x+1)(x^{3}-2)$在有理数域$\mathbb{Q}$上的分裂域。按基础篇的定理4.4.9的证明思路来找出Galois群$Gal(K/\mathbb{Q})$。
\end{example}
