\section{域论*}
\subsection{Galois基本理论}
\original
{
	本节中我们介绍著名的Galois理论的基本定理。这是对整个数学发展起重要推动作用的理论之一。
	\par
	我们仍局限于讨论特征为0的域的情形。可以完全设想我们就是在复数域$\mathbb{C}$中进行的。
	\par
	设$K/F$是$F$上分裂域。这里把下面将用到的域论中或群论中的一些结果重温一下:
	\begin{enumerate}
		\item $F$上分裂域和$F$上正规扩域是一回事;
		\item $F$上分裂域$K=F(\theta)=F(\theta_{i})=F(\theta_{1},\theta_{2},\cdots,\theta_{n})$,其中$\{\theta=\theta_{1},\theta_{2},\cdots,\theta_{n}\}$是$\theta$的$F-$最小多项式$p(x)$的全部根;
		\item $F$上分裂域$K=f(\theta)$,而$\theta$的$F-$最小多项式$p(x)$的次数为$n$,则有
		\begin{equation*}
			n=[K:F]=|Gal(K/F)|;
		\end{equation*}
		\item $F\subseteq L \subseteq K,F,L,K$是域,则有$[K:F]=[K:L][L:F]$;
		\item $\{e\}\subseteq H\subseteq G$,$\{e\}$(单位元组成的群)、H、G是有限群,则有
		\begin{equation*}
			|G|=[G:\{e\}]=[G:H][H:\{e\}]=[G:H]\cdot|G|;
		\end{equation*}
	
		\item 若G是有限群,则对任意$g\in G$有$gG=Gg=G$。
	\end{enumerate}
	取定F上分裂域$K=F(\theta)$,设$\theta$的一个$F-$极小多项式为$p(x)$,$G=Gal(K/F)$。令
	\begin{equation*}
		\begin{aligned}
			&\mathbb{K}=\{K/F\text{的中间域}\,:\, F\subseteq L\subseteq K \} ,\\
			&\mathbb{G}=\{G=Gal(K/F)\text{的一切子群}H\, :\, G\supseteq H\supseteq\{e\}\}.
		\end{aligned}
	\end{equation*}
	\par
	\textbf{\uline{先定义集 $\mathbb{K}$ 到集 $\mathbb{G}$ 的一个对应。}}
	\par
	任取$L\in \mathbb{K}$。
	由$K=F(\theta)\subseteq L(\theta) \subseteq L(\theta_{1},\cdots,\theta_{n}) \subseteq K$得$K=L(\theta)=L(\theta_{1},\cdots,\theta_{n})$,因而$K$是多项式$p(x)\in F[x]\subseteq L[x]$在域$L$上的分裂域。今考虑扩域$L$上分裂域$K$的Galois群$Gal(K/L)$。若$\phi\in Gal(K/L) $,即$\phi $是域$K$的$L-$自同构。$\phi$保持$L$中元素不动,当然更保持$F$中元素不动,因而$\phi$是域$K$的$L-$自同构,即$\phi\in Gal(K/F) $。\uline{随之有 $Gal(K/L)\subseteq Gal(K/F)=G $ 。}这样就得对应
	\begin{equation*}
		Gal:\,\,
		\begin{aligned}
			\mathbb{K}&\rightarrow\mathbb{G}\\
			L&\mapsto Gal(K/L).
		\end{aligned}
	\end{equation*}
	\uline{显然 $F\mapsto Gal(K/F)=G,K\mapsto Gal(K/K)=\{e\}$ 。}
	\par
	\textbf{\uline{其次定义集 $\mathbb{G}$ 到集 $\mathbb{K}$ 的一个对应。}}
	\par
	任取$H\in \mathbb{G}$。把$\alpha\in K $在$F-$自同构$\phi $下的像记作$\alpha\phi $,规定
	\begin{equation*}
		Inv\, H=\{\alpha\in K|\forall\phi\in H,\alpha\phi=\alpha\}.
	\end{equation*}
	\uline{直接验证可知 $Inv\,H$  是 $K$ 的子域且 $F\subseteq Inv\,H$ },即$Inv\, H$是中间域,因而$Inv\, H\in \mathbb{K}$。称$Inv \, H $为$H$的不变子域(也就是$H-$集$K$的不变元组成的子集)。这样就得对应
	\begin{equation*}
		Inv:\,\,
		\begin{aligned}
			\mathbb{G}&\rightarrow\mathbb{K}\\
			H&\mapsto Inv\,H.
		\end{aligned}
	\end{equation*}
	显然 $\{e\}\mapsto Inv\{e\}=K$ 。\uline{我们还将看到 $G\mapsto Inv\, G=F$。}
	\par
	下一步想证的该是:对应Gal和对应Inv是互逆的,即想证:$L\in \mathbb{K}$,$H\in \mathbb{G}$,
	\begin{equation*}
		\begin{aligned}
			\text{Inv}(\text{Gal}(K/L))&=L,\\
			\text{Gal}(K/\text{Inv}H)&=H.
		\end{aligned}
	\end{equation*}
	这就是Galois对应。
}
{P175}

\begin{proposition}
	\begin{itemize}
		\item 验证:$F$上分裂域$K=F(\theta)=F(\theta_{i})=F(\theta_{1},\theta_{2},\cdots,\theta_{n})$,其中$\{\theta=\theta_{1},\theta_{2},\cdots,\theta_{n}\}$是$\theta$的$F-$最小多项式$p(x)$的全部根。
		\item 验证: $Gal(K/L)\subseteq Gal(K/F)=G $ 且$Gal(K/L)\leq Gal(K/F)=G$。
		\item 验证:$Inv\,H$  是 $K$ 的子域且 $F\subseteq Inv\,H$。
		\item 验证:显然 $F\mapsto Gal(K/F)=G,K\mapsto Gal(K/K)=\{e\}$ 。
		\item 验证:显然 $\{e\}\mapsto Inv\{e\}=K$ 。我们还将看到 $G\mapsto Inv\, G=F$(这一点不是显然的)。
	\end{itemize}
\end{proposition}

\begin{proof}
	\begin{itemize}
		\item \unfinished{(反例,$f(x)=x^{3}-2$,$F=\mathbb{Q}$)已知$\theta$是$F$上的代数元,$f(x)\in F[x]$是$\theta$的极小多项式,若$K$是$f(x)$的分裂域,那么有$K=F(\theta)=F(\theta_{i})=F(\theta_{1},\cdots,\theta_{n})$。}
		\par
		若任给一多项式$f(x)\in F[x]$的分裂域$K$,那么$K$是单扩域吗?
	\end{itemize}
\end{proof}









\subsection{一个例子}

\begin{example}
	设K是多项式$f(x)=(x^{2}+x+1)(x^{3}-2)$在有理数域$\mathbb{Q}$上的分裂域。按基础篇的定理4.4.9的证明思路来找出Galois群$Gal(K/\mathbb{Q})$。
\end{example}
