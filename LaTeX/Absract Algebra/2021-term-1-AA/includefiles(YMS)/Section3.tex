%姚慕生,第三章,环论
\section{环论}
\subsection{基本概念}
\subsection{子环、理想与商环}
\subsection{环的同态}
\subsection{整环,分式域}
\subsection{唯一分解环}



\begin{example}
	Z是Gauss整区。	
\end{example}

\begin{example}
	数域上的多项式环是Gauss整区。
\end{example}

\begin{example}
	验证$Z(-\sqrt{5})=\{m+n\sqrt{5}\, \big| m,n\in Z\}$不是Gauss整区。
\end{example}

\begin{proof}
	首先验证$Z(\sqrt{-5})=\{m+n\sqrt{-5}\, \Big| m,n\in Z\}$是一个整区。\par
	假设$Z(\sqrt{-5})$两个数满足$(m_{1}+n_{1}\sqrt{-5})\, \cdot \, (m_{2}+n_{2}\sqrt{-5})=0$,则有:
\begin{equation*}
	\begin{aligned}
			m_{1}\cdot m_{2}-5n_{1} \cdot n_{2}&=0\\
			m_{1}\cdot n_{2}+m_{2}\cdot n_{1}&=0.
	\end{aligned}\tag{-}
\end{equation*}
	\begin{enumerate}
		\item $m_{1}=0$的情形:
		\begin{itemize}
			\item 若$m_{2}=0$,则必有$n_{1}=0$或者$n_{2}=0$。此时两数中必有一个为零。
			\item 若$m_{2}\neq 0$,则必有$n_{1}=0$。此时两数中必有一个为零。
		\end{itemize}
		\item $m_{1}\neq 0$的情形:
		\begin{itemize}
			\item 若$m_{2}\neq 0$,则$m_{1},n_{1},m_{2},n_{2}$均不为零,且有:
			\begin{equation*}
				\frac{m_{1}}{m_{2}}=-\frac{n_{1}}{n_{2}}=k\tag{+}
			\end{equation*}
			将(+)代入(-)可得$m_{1}=n_{1}=0$,产生矛盾。
			\item 由上一点得$m_{2}=0$,由(-)可得$n_{2}=0$。
			
		\end{itemize}
	\end{enumerate}
	则$Z(\sqrt{-5})$为整区,再说明$Z(\sqrt{-5})$不是Gauss整区。\par
	首先指出$Z(\sqrt{-5})$的单位。若$ab=1\, , a,b\in Z(\sqrt{-5})$,则有$|ab|=|a|\cdot|b|=1$。即$Z(\sqrt{-5})$的单位为1或者-1。\par
	考虑$9=3\cdot 3=(2+\sqrt{-5})\cdot(2-\sqrt{-5})$,则$3,2\pm \sqrt{-5}$为不可约元,且互不相伴。\par
	即$Z(\sqrt{-5})$不是Gauss整区(UFD)。
\end{proof}

\subsection{PID与欧式整区}

\begin{example}
	证明:$Z[-1]=\{m+n\sqrt{-1}\ \big| n\in \mathbf{Z}\}$是Gauss整区和欧式整区。
\end{example}

\begin{proof}
	显然,$Z[-1]$为一个整区。
	\begin{enumerate}
		\item 验证$Z[-1]$为一个Gauss整区。
		\begin{itemize}
			\item 
		\end{itemize}
		\item 验证$Z[-1]$为一个欧式整区。考虑映射:
		\begin{equation*}
		\text{\large $\delta:$}\quad
		\begin{aligned}
			Z[-1]^{*} &\rightarrow N^{*}\\
			x(=a+b\sqrt{-1}) &\rightarrow |x|=a^{2}+b^{2} 
		\end{aligned}
		\end{equation*}
		\begin{itemize}
			\item 
		\end{itemize}
	\end{enumerate}
	
		
\end{proof}

\subsection{域上的一元多项式环}
\begin{lemma}\label{Ylm030701}
	设$f(x),g(x)\in F[x]$,则
	\begin{equation*}
		deg(f(x)g(x))=deg \, f(x) +deg \, g(x).
	\end{equation*}
\end{lemma}

\begin{proof}
	
\end{proof}

\begin{corollary}
	$F[x]$是一个整区。
\end{corollary}

\begin{proof}
	由引理\ref{Ylm030701},显然。
\end{proof}

\begin{lemma}\label{Ylm030702}
	给定$F[x]$中两个多项式$f(x),g(x)$,其中$g(x)\neq 0$,则在$F[x]$中存在$q(x)$及$r(x)$,使
	\begin{equation*}
			f(x)=q(x)g(x)+r(x),
	\end{equation*}
	且\hspace{\fill}$deg \, r(x)< deg \, g(x).$\hspace{\fill}{}
\end{lemma}

\begin{theorem}
	$F[x]$是欧式整区。
\end{theorem}

\begin{corollary}
	$F[x]$是PID。
\end{corollary}

\begin{corollary}
	$F[x]$是UFD。
\end{corollary}

\begin{corollary}
	$F[x]$中任意两个多项式$f(x),g(x)$都有最大公因子$d(x)$,且存在$s(x),t(x)$,使
	\begin{equation*}
		d(x)=s(x)f(x)+t(x)g(x).
	\end{equation*}
\end{corollary}
\par
\original{$F[x]$中的不可约元就是不可约多项式,即不再能分解为两个次数低于原多项式之积的多项式,$F[x]$的不可约元就是素元。}{P107}
\begin{proposition}
	\begin{enumerate}
		\item $F[x]$中的不可约元等价于不可约多项式。
		\item $F[x]$中的不可约元等价于素元。
	\end{enumerate}
\end{proposition}
\begin{proof}
	\begin{enumerate}
		\item $''\Rightarrow''$ 若$p(x)\in F[x]$为不可约元,且设$p(x)=p_{1}(x)\, p_{2}(x)$,则$p_{i}(x)\, (i=1,2)$要么为$c$要么为$cp(x)$。则$p(x)$为不可约多项式。\\
		$''\Leftarrow''$ 若$p(x)\in F[x]$为不可约多项式,则若$q(x)\Big| p(x)$,则$q(x)=c$或者$q(x)=cp(x)$,故而$p(x)$为不可约元。
		\item $''\Rightarrow''$ 由于$F[x]$为UFD,所以满足素性条件,即不可约元都为素元。\\
		$''\Leftarrow''$ 任取素元$p(x)\in F[x]$,若$p_{1}(x)\Big| p(x)$,则$\exists \, p_{2}(x)$,满足$p(x)=p_{1}(x)p_{2}(x)$。则有$p(x)\Big| p_{1}(x)p_{2}(x)$,若$p(x)\Big | p_{1}(x)$,则$p_{1}(x)$不是真因子,若$p(x)\Big | p_{2}(x)$,则$p_{1}(x)$为单位,也不是真因子,所以$p(x)$为不可约元。
	\end{enumerate}
\end{proof}
\begin{corollary}
	$F[x]$中任一多项式都可唯一地分解成为有限个不可约多项式的乘积。
\end{corollary}

\begin{lemma}
	$F[x]$中多项式$f(x)$不可约的充要条件是$(f(x))$是$F[x]$的极大理想。即不存在$F[x]$的真理想$I$真包含$(f(x))$。
\end{lemma}

\begin{lemma}
	设$R$是带1的交换环,$I$是$R$的真理想,则$I$是$R$的极大理想的充要条件为$R/I$是一个域。
\end{lemma}

\begin{proof}
	考虑介于$I$和$R$之间的真理想集合$\Sigma=\{J\subset R\, \big| \, I\subsetneq J \subsetneq R ,and \, J \, is \, an \, ideal \}$
	和$R/I$的真理想集合$\Sigma'=\{\{0\}\neq J' \subsetneq R/I \, \big| \, J' \, is \, an \, ideal \}$,可验证$\Sigma$与$\Sigma'$存在一一对应。
	\begin{equation*}
		\text{\large$\phi:$\,\, }
		\begin{aligned}
			\Sigma&\rightarrow\Sigma'\\
			J&\mapsto J'=\{x+I \, \big| \, x\in J \}
		\end{aligned}
	\end{equation*}
	容易验证$J'$是一个非零真理想(依次验证,$J'$是一个理想,且非零的,且为真)。即$\phi$是合理定义的。再验证$\phi$为双射。
	\begin{enumerate}
		\item 单射。反证法,若$J_{1}\neq J_{2}$,而$J'_{1}=J'_{2}$,那么$\exists \, x_{1}\in J_{1},x_{1}\not\in J_{2},and \, x_{2}\in J'_{2}$使得$x_{1}+I=x_{2}+I $,即$ x_{1}-x_{2}\in I\subset J_{2}$,从而$x_{1}\in J_{2}$,产生矛盾。
		\item 满射。$\forall \, J'\in \Sigma' $,考虑$J=\{x\in R\,\big| \, x+I\in J'\}$,容易验证$J$为一个理想,且满足$I\subsetneq J\subsetneq R$。
	\end{enumerate}
	\par
	“$\Rightarrow$”,$I$是极大理想说明,$\Sigma$是空集,所以$\Sigma'$也是空集。所以$R/I$只有平凡理想,所以$R/I$为域。
	\par
	“$\Leftarrow$”,$R/I$为域,说明$\Sigma'$为空集,所以$\Sigma$为空集,所以$I$为极大理想。
\end{proof}

\begin{corollary}
	$F[x]$中多项式$f(x)$不可约的充要条件为$F[x]/(f(x))$是一个域。
\end{corollary}

\begin{example}
	设$F$是有理数域,$p(x)=x^{3}-2$。
\end{example}

\begin{example}
	设$F$是最简单的域$F=Z_{2}=\{0,1\}$,$p(x)=x^{2}+x+1$是$F$上的不可约多项式(它在$F$中没有根),求$F[x]/(p(x))$的全部元素。
\end{example}
\begin{proof}
	
\end{proof}

\original
{
	现在假定$S$是域$F$的扩张,即$F$是$S$的子环($F$本身是一个域)且$F$的恒等元1就是$S$的恒等元,设$S$也是一个交换环,$u$是$S$中的一个元素,令$F[u]=\{a_{0}+a_{1}u+a_{2}u^{2}+a_{3}u^{3}+...+a_{n}u^{n}\,\big| \, a_{i}\in F,n\in N\}$,考虑映射
	\begin{equation*}
		\text{\Large $\eta:$}\,\,
		\begin{aligned}
			F[x]&\rightarrow F[u]\\
			a_{0}+a_{1}x+...+a_{n}x^{n}&\rightarrow a_{0}+a_{1}u+...+a_{n}u^{n}
		\end{aligned}
	\end{equation*}
不难验证$F[u]$是$S$的子环且$\eta$是$F[x]$到$F[u]$的映上同态。如果$\eta$是一个同构,则称$u$是$F$上的一个超越元。若$Ker \, \eta \neq 0$,则称$u$是$F$上的代数元。当$u$是超越元时,$a_{0}+a_{1}u+...+a_{n}u^{n}=0$当且仅当所有的$a_{i}=0$。当$u$是代数元时,$Ker \, \eta$可由一个多项式$f(x)$生成,这时$F[u]\,\cong\, F[x]/(f(x))$,且$f(u)=0$,即$u$适合$F$上的一个一元多项式。如$f(x)$是一个不可约多项式,则$F[u]$是一个域。代数元,超越元这种定义方式是经典定义的自然推广。
}{P108}
\begin{proposition}
	\begin{enumerate}
		\item 验证$F[u]$是$S$的子环且$\eta$是$F[x]$到$F[u]$的映上同态。
		\item 说明当$u$是代数元的时候,$Ker\, \eta$可由一个多项式$f(x)$生成,这时$F[u]\cong \, F[x]/(f(x))$。
	\end{enumerate}
\end{proposition}
\begin{proof}
	
\end{proof}

\original
{
	还可以定义多项式根的概念。设$f(x)\in F[x]$,$S$中的元$u$如果适合$f(u)=0$,就称它为$f(x)$的根。若$u$是$F$上的代数元,则必有一个次数最小的首一多项式$f(x)$,使$u$是它的根,这个多项式称为代数元$u$的最小多项式。最小多项式是唯一确定的。
}
{P108}

\begin{proposition}
	\begin{enumerate}
		\item 说明最小多项式的存在性,唯一性。
		\item 为什么不定义超越元的最小多项式?
	\end{enumerate}
\end{proposition}

\begin{proof}
	
\end{proof}

\begin{lemma}[余数定理]
	若$f(x)\in F[x]$,$a\in F$,则存在唯一的$q(x)$,使
	\begin{equation*}
		f(x)=(x-a)q(x)+f(a).
	\end{equation*}
\end{lemma}
\begin{proof}
	由引理\ref{Ylm030702}显然。
\end{proof}

\begin{corollary}
	$(x-a)\big| f(x) $的充要条件是$a$为$f(x)$的根。
\end{corollary}

\begin{theorem}
	$F[x]$上的$n$次多项式$f(x)(n>0)$在域$F$中最多只有$n$个不同的根。
\end{theorem}

\begin{theorem}
	任意一个域的乘法群的有限子群是循环群。
\end{theorem}
\begin{proof}
	利用定理\ref{Yth020403}。设$G$是某个域$F$的乘法群的有限子群,$m$是使$\forall a\in G$,满足$a^{m}=1$的最小的正整数。\par
	首先有,$a^{|G|}=1$,则得$m\leq |G|$。\par
	再考虑多项式$f(x)=x^{m}-1$,$f(x)$在$F$中最多有$m$个不同的根,所以$|G|\leq m$。\par
	所以$|G|=m$。
\end{proof}
\begin{corollary}
	任一有限域的乘法群是一个循环群。
\end{corollary}

\original
{
	比如$Z_{p}$(p是素数)的乘法群是阶为$p-1$的循环群。
}{109}
\begin{proposition}
	\begin{enumerate}
		\item $Z_{p}$为域$\Leftrightarrow$ $p$为素数。
		\item 此循环群的生成元是什么?
	\end{enumerate}
\end{proposition}

\subsection{交换环上的多项式环}

\subsection{素理想}

本节涉及的环都假定为含恒等元的交换环。

\begin{definition}
	设$P$是环$R$的理想且$P\neq R$,若对$R$中元素$a,b$从$ab\in P$可推出$a\in P$或者$b\in P$,则称$P$是环$R$的素理想。(素理想在代数几何中有非常重要的应用。)
\end{definition}

\begin{theorem}
	环$R$的理想$P$是素理想的充要条件是$R/P$是整区。	
\end{theorem}
\begin{proof}
	“$\Rightarrow$”,$R/P$显然是含幺交换环,再利用$P$是素理想证明$R/P$是整环即可。
	\par
	“$\Leftarrow$”,根据$R/P$是整区,可验证$P$是素理想。
\end{proof}

\begin{corollary}
	环$R$的极大理想必是素理想。
\end{corollary}
\begin{proof}
	若$P$是$R$的极大理想,则$R/P$是域。
\end{proof}

\begin{theorem}
	设$R$是含恒等元的交换环,$I$是它的一个理想且$I\neq R$,则$I$一定包含在$R$的某个极大理想之中。
\end{theorem}

\begin{corollary}
	任意一个含恒等元的环都有极大理想。
\end{corollary}

\begin{corollary}
	$R$中的任一不可逆元素均含在某个极大理想之中。
\end{corollary}

\begin{definition}
	交换环$R$所有素理想的交是$R$的理想,称为$R$的小根。
\end{definition}

\begin{theorem}
	交换环$R$的小根等于$R$的所有幂零元素和零元素组成的理想即$R$的诣零根。
\end{theorem}

\begin{definition}
	交换环$R$的所有极大理想的交$J$是一个理想,称为$R$的大根或Jacobson根。
\end{definition}

\begin{theorem}
	环$R$中元素$a$属于大根$J$的充要条件是对任意的$r\in R$,元素$1-ar$是可逆元。
\end{theorem}

\begin{example}
	设$\mathbb{Z}$是整数环,$\mathbb{Z}$是PID,$\mathbb{Z}$中理想$(m)$是素理想的充要条件是$m$是素数。这时$(m)$也是极大理想。
\end{example}
\begin{proof}
	先证$\mathbb{Z}$是PID。
	\par
	“$\Rightarrow$”,任取$a,b\in \mathbb{Z}$,若$m\big| ab$,则$ab\in (m)$,由于$(m)$是素理想,所以有$a\in (m),or\, b\in (m)$,即$m\big| a,or\, m\big| b$,所以$m$为素数。
	\par
	“$\Leftarrow$”,任取$a,b\in \mathbb{Z}$,若满足$ab\in (m)$,则$m\big| ab$,因为$m$是素数,所以$m\big| a,or\, m\big| b$,即$a\in (m),\, or\, b\in (m)$,所以$(m)$为素理想。
\end{proof}

\begin{example}
	设$F[x]$是域$F$上的多项式环,$F[x]$中理想$I=(f(x))$是素理想当且仅当$f(x)$是不可约多项式。事实上,这时$F[x]/(f(x))$是域,因此$I$是极大理想。
\end{example}

\begin{example}
	设$R$是PID,则$R$中理想$(a)$是素理想的充要条件是$a$是素元。这时$(a)$也是$R$的极大理想。事实上,$R/(a)$是一个域。
\end{example}

\begin{example}
	设$R=\mathbb{Z}[x]$是整数多项式环。则理想$(2,x)$是极大理想。因而也是素理想。理想$(x)$是素理想而不是极大理想。
\end{example}

\begin{proof}
	(\textbf{法一(from 刘绍学P90):})取$I\supsetneq (2,x)$,由于$(2,x)=2h(x)+xg(x),h(x)\in F[x],g(x)\in F[x]$,所以$\exists \, m(x)\in I,and\, x\not\in (2,x)$,即$m(x)$的常数项为奇数,从而$1\in I$,所以$I=R$。
	\par
	任取$f(x)g(x)\in (x)$,则$x\big| f(x)g(x)$,所以$x\big| f(x),or\, x\big| g(x)$。从而$(x)$是素理想。又$(x)\subsetneq (2,x)$,所以$(x)$不是极大理想。
	\par
	(\textbf{法二(from 姚慕生P117):})
	可验证$R/(x)\cong\mathbb{Z}$,$\mathbb{Z}$是整区不是域,所以$(x)$为素理想。再验证$R/(2,x)\cong \mathbb{Z}_{2}$,$\mathbb{Z}_{2}$为一个域,所以$(2,x)$为极大理想。
\end{proof}

\begin{proposition}
	\begin{enumerate}
		\item 设$P_{1},\cdots,P_{n}$是环$R$的$n$个素理想,$I$是$R$的理想。已知$I\subseteq\bigcup_{i=1}^{n}P_{i}$,则必存在某个$1\leq i\leq n$适合$I\subseteq P_{i}$。
		\item 设$I_{1},\cdots,I_{n}$是$R$的$n$个理想,$P$是包含$\bigcap_{i=1}^{n}I_{i}$的素理想,则存在某个$i$,$I_{i}\subseteq P$。又若$P=\bigcap_{i=1}^{n}I_{i}$,则存在某个$i$,$P=I_{i}$。
	\end{enumerate}
\end{proposition}

\begin{proof}
	\begin{enumerate}
		\item 即证$ \forall \, 1\leq i\leq n $,$I\subsetneq P_{i}$,则有$I\subsetneq\bigcup_{i=1}^{n}P_{i}$。
		\par
		当$n=1$时,结论成立。
		\par
		现假设$n-1$的情形成立,证$n$的情形成立。对于$P_{1},\cdots,P_{i-1},P_{i+1},\cdots,P_{n}$这$n-1$个素理想,由归纳假设,$\exists \, a_{i}\in I,and \, a_{i}\not\in P_{1},P_{2},\cdots,P_{i-1},P_{i+1},\cdots,P_{n}$,若$a_{i}\not\in P_{i}$,则$I\subsetneq \bigcup_{i=1}^{n}P_{i}$,结论成立。若$a_{i}\in P_{i}$,考虑
		\begin{equation*}
			b=a_{2}a_{3}\cdots a_{n}+a_{1}a_{3}\cdots a_{n}+\cdots +a_{1}a_{2}\cdots a_{n-1}
		\end{equation*}
		则$b\in I,and \, b\not\in \bigcup_{i=1}^{n}P_{i}$。
		\par
		所以结论成立。
		\item 若$\forall \, i,I_{i}\subsetneq P$,则$\exists \, a_{i}$,满足$a_{i}\in I_{i},and \, a_{i}\not\in P$,考虑$b=a_{1}a_{2}\cdots a_{n}\in \bigcap_{i=1}^{n}I_{i}$,由$P$是素理想,所以$\exists a_{i},a_{i}\in P$,产生矛盾。
	\end{enumerate}
\end{proof}
