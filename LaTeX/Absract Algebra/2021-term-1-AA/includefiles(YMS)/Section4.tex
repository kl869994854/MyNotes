%姚慕生,第四章,域与Galois理论
\section{域与Galois理论}

\subsection{域的扩张}

\original
{
	我们已经知道,任何一个域都包含一个“最小子域”。在域的特征为零时,这个子域同构于有理数域;在域的特征为$p$时,这个子域同构于模$p$的剩余类域$Z_{p}$,这样的最小子域称为\textbf{素域}。因此,任何一个域都是其素域的扩张。研究域的一个方法便是从一个较“小”的域出发来构造“大”的域,即它的扩域。
}
{P125}
\begin{proposition}
	\begin{itemize}
		\item 任何一个域都包含一个“最小子域”。
		\item 在域的特征为零时,这个子域同构于有理数域;在域的特征为$p$时,这个子域同构于模$p$的剩余类域$Z_{p}$。
	\end{itemize}
\end{proposition}

\begin{hint}
	考虑乘法单位元生成的子域。
\end{hint}

\original
{
	假定$E$是$F$的扩域,$S$是$E$的子集,记$F(S)$为$E$的由$F$及$S$生成的子域,即$E$的所有包含$F$同时也包含$S$的子域之交。若$T$是$E$的又一个子集,则$F(S)(T)=F(S\cup T)$。原因很简单,因为等式两边都表示$E$的所有包含$F,S,T$的子域之交。特别,当$S$是有限集时,比如$S=\{u_{1},u_{2},\cdots,u_{n}\}$,有
	\begin{equation*}
		F(u_{1},\cdots,u_{k})=F(u_{1},\cdots,u_{k-1})(u_{k}),\,\, k=2,3,\cdots ,n,
	\end{equation*}
	如果$S=\{u\}$,即$S$是单点集,则称$F(u)$为$F$的单扩张,元素$u$称为本原元。单扩张是最简单的扩张。
}
{P125}

\begin{proposition}
	\begin{itemize}
		\item $F(S)$长什么样子?
		\item 验证:$F(S)(T)=F(S\cup T)$。
	\end{itemize}
\end{proposition}

\begin{proof}
	\begin{enumerate}
		\item\begin{itemize}
			\item $S=\{a\}$
			\item $S=\{a_{1},\cdots,a_{n}\}$
			\item 任意$S\in E$		      \end{itemize}
		\item 显然。
	\end{enumerate}
\end{proof}

\original
{
	设$x$是未定元,$F[x]$是$F$上的多项式环,记$F[u]$为$E$中$u$的多项式全体,即形如$a_{0}+a_{1}u+\cdots+a_{n}u^{n}(a_{i}\in F)$的元素全体组成的子集,则不难看出,$F[u]$是$E$的子环。现作$F[x]\rightarrow F[u]$的同态映射$\phi$:
	\begin{equation*}
		\phi(f(x))=f(u),
	\end{equation*}
	则$\phi$是映上的环同态。由第三章多项式环的理论知道,当$u$是$F$上的代数元时,$Ker\,\phi=(g(x))=(g(x))$,$g(x)$是$F$上的某个不可约多项式,且$F[x]/(g(x))\cong F[u]$,这时$F[x]/(g(x))$是域,因此$F[u]$是$E$的子域,于是$F[u]=F(u)$。$F(u)$中的元素可以写成为次数低于$deg\, g(x)$的$u$的多项式。当$u$是$F$上的超越元时,$F[x]\cong F[u]$,这时$F[u]$不是$E$的子域,显然$F(u)$是$F[u]$的分式域,即$F(u)=\{\frac{f(u)}{g(u)}\big| f,g\in F[u],g\neq 0\}$。
}
{P125}

\begin{proposition}
	\begin{itemize}
		\item 当$u$是代数元时,$F[u]=F(u)$。
		\item $F(u)$中的元素可以写成为次数低于$deg\, g(x)$的$u$的多项式。
		\item 当$u$是超越元时,$F(u)$是$F[u]$的分式域。
	\end{itemize}
\end{proposition}

\begin{proof}
	\begin{enumerate}
		\item 
		\item 因为$F(u)=F[u]\cong F[x]/(g(x))$,$\forall f(x)\in F[x]$,有
		\begin{equation*}
			f(x)=q(x)g(x)+r(x),\,\,(deg\,r(x)< deg\, g(x)),
		\end{equation*}
		即$f(x)+(g(x))=r(x)+(g(x))$,$r(u)$即为符合条件的多项式。
		\item 
	\end{enumerate}
\end{proof}



