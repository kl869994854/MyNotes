%姚慕生,第四章,域与Galois理论
\section{域与Galois理论}

\subsection{域的扩张}

\original
{
	我们已经知道,任何一个域都包含一个“最小子域”。在域的特征为零时,这个子域同构于有理数域;在域的特征为$p$时,这个子域同构于模$p$的剩余类域$Z_{p}$,这样的最小子域称为\textbf{素域}。因此,任何一个域都是其素域的扩张。研究域的一个方法便是从一个较“小”的域出发来构造“大”的域,即它的扩域。
}
{P125}
\begin{proposition}
	\begin{itemize}
		\item 任何一个域都包含一个“最小子域”。
		\item 在域的特征为零时,这个子域同构于有理数域;在域的特征为$p$时,这个子域同构于模$p$的剩余类域$Z_{p}$。
	\end{itemize}
\end{proposition}

\begin{hint}
	考虑乘法单位元生成的子域。
\end{hint}

\original
{
	假定$E$是$F$的扩域,$S$是$E$的子集,记$F(S)$为$E$的由$F$及$S$生成的子域,即$E$的所有包含$F$同时也包含$S$的子域之交。若$T$是$E$的又一个子集,则$F(S)(T)=F(S\cup T)$。原因很简单,因为等式两边都表示$E$的所有包含$F,S,T$的子域之交。特别,当$S$是有限集时,比如$S=\{u_{1},u_{2},\cdots,u_{n}\}$,有
	\begin{equation*}
		F(u_{1},\cdots,u_{k})=F(u_{1},\cdots,u_{k-1})(u_{k}),\,\, k=2,3,\cdots ,n,
	\end{equation*}
	如果$S=\{u\}$,即$S$是单点集,则称$F(u)$为$F$的单扩张,元素$u$称为本原元。单扩张是最简单的扩张。
}
{P125}

\begin{proposition}
	\begin{itemize}
		\item $F(S)$长什么样子?
		\item 验证:$F(S)(T)=F(S\cup T)$。
	\end{itemize}
\end{proposition}

\begin{proof}
	\begin{enumerate}
		\item\begin{itemize}
			\item $S=\{a\}$
			\item $S=\{a_{1},\cdots,a_{n}\}$
			\item 任意$S\in E$		      \end{itemize}
		\item 显然。
	\end{enumerate}
\end{proof}

\original
{
	设$x$是未定元,$F[x]$是$F$上的多项式环,记$F[u]$为$E$中$u$的多项式全体,即形如$a_{0}+a_{1}u+\cdots+a_{n}u^{n}(a_{i}\in F)$的元素全体组成的子集,则不难看出,$F[u]$是$E$的子环。现作$F[x]\rightarrow F[u]$的同态映射$\phi$:
	\begin{equation*}
		\phi(f(x))=f(u),
	\end{equation*}
	则$\phi$是映上的环同态。由第三章多项式环的理论知道,当$u$是$F$上的代数元时,$Ker\,\phi=(g(x))=(g(x))$,$g(x)$是$F$上的某个不可约多项式,且$F[x]/(g(x))\cong F[u]$,这时$F[x]/(g(x))$是域,因此$F[u]$是$E$的子域,于是$F[u]=F(u)$。$F(u)$中的元素可以写成为次数低于$deg\, g(x)$的$u$的多项式。当$u$是$F$上的超越元时,$F[x]\cong F[u]$,这时$F[u]$不是$E$的子域,显然$F(u)$是$F[u]$的分式域,即$F(u)=\{\frac{f(u)}{g(u)}\big| f,g\in F[u],g\neq 0\}$。
}
{P125}

\begin{proposition}
	\begin{itemize}
		\item 当$u$是代数元时,$F[u]=F(u)$。
		\item $F(u)$中的元素可以写成为次数低于$deg\, g(x)$的$u$的多项式。
		\item 当$u$是超越元时,$F(u)$是$F[u]$的分式域。
	\end{itemize}
\end{proposition}

\begin{proof}
	\begin{enumerate}
		\item 
		\item 因为$F(u)=F[u]\cong F[x]/(g(x))$,$\forall f(x)\in F[x]$,有
		\begin{equation*}
			f(x)=q(x)g(x)+r(x),\,\,(deg\,r(x)< deg\, g(x)),
		\end{equation*}
		即$f(x)+(g(x))=r(x)+(g(x))$,$r(u)$即为符合条件的多项式。
		\item 
	\end{enumerate}
\end{proof}

\original
{
	通过以上分析我们可以看出:第一,给定$F$及$u$,$F(u)$完全可以构造出来;第二,单扩张可以分为两种不同的类型。对一般的域扩张,我们也有两种类型,即代数扩张与超越扩张。
}
{P126}
\begin{definition}
	设$E$是$F$的扩域,若$E$中每个元都是$F$上的代数元,则称$E$是$F$的代数扩域或代数扩张。不然,就称$E$是$F$的超越扩张或超越扩域。
\end{definition}

\begin{example}
	$F=\mathbb{Q}$,即有理数域,$u=\sqrt[3]{2},E=\mathbb{R}$,为实数域,则$F(u)=Q(\sqrt[3]{2})=\{a+b\sqrt[3]{2}+c\sqrt[3]{4}\big| a,b,c\in \mathbb{Q}\}$。
\end{example}

\begin{proof}
	$u=\sqrt[3]{2}$为$F$上的代数元,所以$F(u)=F[u]$。
\end{proof}

\begin{example}
	$F=\mathbb{Q},E=\mathbb{C}$,即复数域,$u=\sqrt{-1}$,则
	\begin{equation*}
		\mathbb{Q}(\sqrt{-1})=\{a+b\sqrt{-1}\big| a,b\in F\}
	\end{equation*}
\end{example}

\begin{proof}
	$\sqrt{-1}$是$F$上的代数元,所以$F(\sqrt{-1})=F[u]$。
\end{proof}

\begin{example}
	$F=\mathbb{Q},E=\mathbb{R},u=\pi$,由于$\pi$是超越数,因此$\mathbb{Q}(\pi)=\{\frac{f(\pi)}{g(\pi)}\big| f,g\in Q[x]\,and\,g\neq 0\}$
\end{example}

\original
{
	我们再从另外一个角度来考察域的扩张问题。设$E$是$F$的扩域,则$E$可以看成是$F$上的线性空间,$E$中元素作为向量。向量加法即是$E$作为域的加法,$F$中元素对$E$中向量的纯量乘法即等于$E$中元素的乘法(注意$F$是$E$的子集)。通常记$[E:F]$为$E$作为$F$上线性空间的维数。如果$[E:F]<\infty$,则称$E$是$F$的有限维扩张或简称为有限扩张;若$[E:F]=\infty$,则称$E$是$F$的无限维扩张或称无限扩张。
}
{P126}
\par 若$E$是$F$的扩域,$K$是$E$的扩域,则有下列重要的维数公式。
\begin{theorem}
	$[K:F]$有限的充要条件是$[K:E]$与$[E:F]$皆有限,这时
	\begin{equation*}
		[K:F]=[K:E]\,[E:F]
	\end{equation*}
\end{theorem}

\begin{proof}
	
\end{proof}


\begin{corollary}
	若$F\subset E\subset K$,且$[K:F]<\infty$,则$[E:F]$及$[K:E]$都是$[K:F]$的因子。特别,若$[K:F]$为素数时,则在$K$与$F$之前没有其他的子域。
\end{corollary}

\original
{
	单代数扩域与有限扩域有着密切的关系。设$E$是$F$的扩域,$u$是$F$上的代数元,则$u$必适合一个$F$上的首一多项式$g(x)$。若$g(x)$是这类多项式中次数最低者,则称之为$u$的最小多项式或$u$的极小多项式。$u$的极小多项式实际上就是我们前面谈到的同构式$F[x]/(g(x))\cong F[u]$中的$g(x)$(只要除以一个$F$中元素,即可使$g(x)$变成为首项系数为1的多项式)。事实上,由$\phi(g(x))=0$即知$g(u)=0$。又若$h(x)$是$u$的极小多项式,则由多项式除法,将$g(x)=h(x)q(x)+r(x)$代入$u$得$r(u)=0$。而$r(x)$的次数比$h(x)$更低,因此必有$r(x)=0$,否则将与$h(x)$的极小性相矛盾。这表明$g(x)=h(x)q(x)$。但$g(x)$是不可约多项式,因此$g(x)=c\,h(x),c\in F$。现在由于$g,h$都是首一多项式,故$g(x)=h(x)$。从这里我们得出一个结论:\uline{代数元\text{$u$}的极小多项式必是不可约的。}有了极小多项式的概念,我们可以证明如下结果:
}
{P127}
\begin{remark}
	这里实际上证明了极小多项式是唯一的。
\end{remark}
\begin{proposition}
	验证:代数元$u$的极小多项式必是不可约的。
\end{proposition}

\begin{proof}
	若$g(x)=g_{1}(x)g_{2}(x)$,$g_{1}(x),g_{2}(x)$都是$g(x)$的真因子。由于$g(u)=g_{1}(u)g_{2}(u)=0$,则必有$g_{1}(u)=0\, or\, g_{2}(u)=0$,与$g_{x}$为极小多项式矛盾。
\end{proof}


\begin{theorem}
	设$E$是$F$的扩域,若$u$是$E$中的元素且是$F$上的代数元,其极小多项式为$g(x)$,则$[F(u):F]$有限且等于$deg \, g(x)$。反之,若$[F(u):F]<\infty$,则$u$必是$F$上的代数元。
\end{theorem}

\begin{corollary}
	有限扩张必是代数扩张。
\end{corollary}

\begin{theorem}[Steinitz定理]
	设$E$是$F$的扩域且$[E:F]<\infty$,则$E=F(u)$的充要条件是$E$与$F$之间只有有限个中间域。
\end{theorem}


\subsection{代数扩域}

\begin{theorem}\label{YMSthe040201}
	设$E$是$F$的扩域,$K$是$E$中所有$F$上代数元的全体组成的集,则$K$是$E$的子域。
\end{theorem}

\begin{proof}
	$\forall u,v\in E$,若$u,v$为代数元,则$[F(u,v):F]=[F(u,v):F(u)]\,[F(u):F]<\infty$,所以$[F(u-v):F],[F(u\,v):F],[F(u^{-1}):F]<\infty$。
\end{proof}

\begin{corollary}
	两个代数数的和、差、积、商仍是代数数。
\end{corollary}

\original
{
	虽然单代数扩域必然是有限扩张,但一般来说,代数扩域不必是有限扩张。比如有理数域上的代数元全体即代数数全体是有理数域的代数扩张,但不是有限扩张。事实上对任意的$n$,由$Eisenstenin$判别法知,有理数域上的$n$次多项式$x^{n}-2$是不可约的,\uline{因此代数数域在有理数域上的维数不可能是有限维的}。
}
{P129}

\begin{proposition}
	验证:代数数域在有理数域上的维数不可能是有限维的。
\end{proposition}

\begin{proof}
	由方程$x^{n}-2=0$找出无穷个线性无关的代数元。
\end{proof}

\begin{theorem}
	设$E$是$F$的扩域,则下列命题等价:
	\begin{enumerate}
		\item[(1)] [E:F]$<\infty$;
		\item[(2)] 存在$E$中有限个代数元$u_{1},u_{2},\cdots u_{n}$,使$E=F(u_{1},u_{2},\cdots,u_{n})$,此时,$E$必是$F$的代数扩域。
	\end{enumerate}
\end{theorem}


代数扩域由传递性,即下述定理成立。
\begin{theorem}\label{YMSthe040203}
	若$E$是$F$的代数扩域,$K$是$E$的代数扩域,则$K$是$F$的代数扩域。
\end{theorem}

\begin{corollary}
	设$E$是$F$的扩域,$K$是$E$中$F$上代数元全体组成的子域,则任何$E$中$K$上的代数元仍数域$K$。
\end{corollary}

\begin{proof}
	$E$中$K$上的代数元全体为$K$的代数扩域,同时也为$F$的代数扩域,所以$E$中$K$上的代数元全体包含于$K$。
\end{proof}

\begin{definition}\label{YMSdef040201}
	设$E$是$F$的扩域,$K$是$E$的子域又是$F$的扩域,若$K$是$F$的代数扩域且任何$K$在$E$中的代数扩域均与$K$重合(即$K$在$E$中无真代数扩张),则称$K$是$F$在$E$中的\textbf{代数闭包}。$K$也成为在$E$中是\textbf{代数封闭的}。
\end{definition}

\original
{
	由定理\ref{YMSthe040201}和定理\ref{YMSthe040203}可知,域$F$上代数元全体构成的$E$的子域就是$F$在$E$中的代数闭包。应当注意,上面的代数封闭概念是一个相对的概念,通常我们所说的代数闭域和代数闭包的定义如下所述。
}
{P130}

\begin{proposition}
	域$F$上代数元全体构成的$E$的子域就是$F$在$E$中的代数闭包。
\end{proposition}

\begin{proof}
	按照代数闭包的定义验证,显然成立。
\end{proof}

\begin{definition}\label{YMSdef040202}
	设$K$是一个域,如果$K$无真代数扩张,则称$K$是一个代数闭域。
\end{definition}

\begin{definition}\label{YMSdef040203}
	设$K$是$F$的扩域,如果$K$是$F$的代数扩域且$K$是一个代数闭域,则称$K$是$F$的代数闭包。
\end{definition}

\begin{theorem}
	设$K$是一个域,下列命题等价:
	\begin{enumerate}
		\item $K$是代数闭域;
		\item $K[x]$中任一不可约多项式的次数等于1。
		\item $K[x]$中任一次数大于零的多项式可分解为一次因子的乘积;
		\item $K[x]$中任一次数大于零的多项式都在$K$中至少有一个根。
	\end{enumerate}
\end{theorem}

\begin{proof}
	2,3,4容易验证是等价的。
	\par
	(2,3,4)$\Rightarrow$(1),任取$K$的代数扩域记为$K_{1}$,$\forall \alpha\in K_{1}$,$\exists f(x)\in K[x]$,使得$f(\alpha)=0$,由于$f(x)$可分解为一次因子的乘积,所以$\alpha\in K$。
	\par
	(1)$\Rightarrow$(2,3,4),\uline{设 $g(x)\in K[x]$ 是一个不可约多项式,则 $g(x)$ 决定了一个 $K$ 的代数扩域 $E$ 且 $[E:K]=deg\, g(x)$ 。  }
\end{proof}

\original
{
	由代数基本定理知,复数域$\mathbb{C}$是一个代数闭域。
}
{P131}

\begin{proposition}
	验证:复数域$\mathbb{C}$是一个代数闭域。
\end{proposition}


\begin{theorem}
	设$K$是代数闭域,$F$是其子域,则$F$在$K$中的代数闭包$\overline{F}$一定是代数闭域,$\overline{F}$也是定义\ref{YMSdef040203}意义下$F$的代数闭包。
\end{theorem}

\begin{proof}
	$\overline{F}$的代数元一定是$K$的代数元。所以$\overline{F}$的任意代数扩张$H$一定满足$H\subset K$,又由$\overline{F}$的定义,$H\subset \overline{F}$,所以$\overline{F}$为代数闭域。
\end{proof}

\original
{
	由此可知,代数数全体是一个代数闭域,它也可以看成是有理数域的代数闭包。
}
{P131}

\begin{theorem}
	任何一个域都有一个代数闭域作为它的扩域,从而任何一个域都有代数闭包。
\end{theorem}


\begin{theorem}
	域$F$的两个代数闭包设为$E_{1}$及$E_{2}$,则必存在同构映射$\phi:E_{1}\rightarrow E_{2}$,使得$\phi(a)=a$对一切$a\in F$成立。
\end{theorem}

\subsection{尺规作图问题}

\subsection{分裂域}



















